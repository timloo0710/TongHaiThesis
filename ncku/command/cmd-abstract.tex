%
% This file is part of ncku-thesis-templete.
%
% ncku-thesis-templete is distributed in the hope that it will be useful,
% you can redistribute it and/or modify
% it under the terms of the Attribution-NonCommercial-ShareAlike
% 4.0 International.
%
% You should have received a copy of the
% Attribution-NonCommercial-ShareAlike 4.0 International
% along with ncku-thesis-templete.
%
% If not, see <http://creativecommons.org/licenses/by-nc-sa/4.0/legalcode.txt>.
%

% Some helper function use in abstract

% ----------------------------------------------------------------------------

% Abstract
\newcommand{\StartChiAbstract}
{
  \StartNewPage

  % Add to "Table of Contents"
  \addcontentsline{toc}{chapter}{中文摘要}

  % Title
  %\centerline{\Large \textbf{摘要}}
\begin{minipage}[c][5cm][t]{\textwidth}
  論文名稱:社群組織之重要構成因素探討:  成員分享行為及\\ 
  \indent \hspace{5.0em} 輔助分享資訊工具研究  ---以g0v零時政府零時政府為例 \\
  校所名稱:\GetSchoolDeptChiName \\
  畢業時間:2015年 6月 \\
  研究生  : \GetAuthorChiName \hspace{12.0em} 指導教授: \GetAdvisorChiNameA \\
  
  %
  %\makebox[4.0em][r]{校所名稱:} %abstract
  %\makebox[6.0em][l]{ \GetSchoolDeptChiName } \\
  %\vspace{0.5cm}  
  %\hspace{2.0em}
  %\makebox[4.0em][r]{畢業時間:}
  %\makebox[6.0em][l]{2015年 6月 } \\
  \vspace{0.5cm}  
  %\hspace{2.0em}
  %\makebox[4.0em][r]{研究生:}
  %\makebox[6.0em][l]{\GetAuthorChiName }  
  \vspace{0.5cm}  
  \makebox[8.0em][c]{}
  %\makebox[4.0em][r]{指導教授:}
  %\makebox[8.0em][l]{\GetAdvisorChiNameA}
  
  \hspace{2.0em}
  
  
\end{minipage}  

  \leftline{\Large \textbf{中文摘要:}}
  
  
  
} % End of \newcommand{}

\newcommand{\StartAbstract}
{
  \StartNewPage

  % Add to "Table of Contents"
  \addcontentsline{toc}{chapter}{英文摘要}

  % Title
\begin{minipage}[c][5cm][t]{\textwidth}
  
  Title of Thesis  :The most important factor of psudo community :Majoring in \\
  \indent \hspace{6.5em} sharing behavior and tools ---Let g0v as example \\
  Name of Institute:Tunghai University  Executive Master of \\
  \indent \hspace{6.5em} Business Administration Program \\
  Graduation Time  :06/2015 \\
  Student Name     : \GetAuthorChiName \hspace{12.0em} Advisor Name: \GetAdvisorChiNameA \\

\end{minipage} 
  
  \leftline{\Large \textbf{Abstract:}}
} % End of \newcommand{}

\newcommand{\EndChiAbstract}
{
  % Keyword
  \AbstractKeyword

  \EndOfPage
} % End of \newcommand{}

\newcommand{\EndAbstract}
{
  % Keyword
  \ExtendedAbstractKeyword

  \EndOfPage
} % End of \newcommand{}

% Extended Abstract
\newcommand{\StartExtendedAbstract}
{
  \StartNewPage

  % Set page
  \baselineskip=20pt
  \setlength{\parindent}{0.0pt}

  % Set style
  \pagestyle{empty}

  % 設定段落之間的距離
  \setlength{\parskip}{0.5cm}

  % Add to "Table of Contents"
  \addcontentsline{toc}{chapter}{Extended Abstract}

  % Add title
  \parbox{\textwidth}{\center \large \textbf{\eTitle}}
  \vspace{0.5cm}

  % Add name
  \centerline{\large \GetAuthorEngName}

  % Add names
  \centerline{Prof. \thinspace \GetAdvisorEngNameA}
  \ifthenelse{\equal{\GetAdvisorEngNameB}{\empty}}
    {}
    {\centerline{Prof. \thinspace \GetAdvisorEngNameB}}

  \ifthenelse{\equal{\GetAdvisorEngNameC}{\empty}}
    {}
    {\centerline{Prof. \thinspace \GetAdvisorEngNameC}}

  % Add department
  \centerline{\GetDeptEngName}
  \centerline{\GetCollEngName}
} % End of \newcommand{}

\newcommand{\EndExtendedAbstract}
{
  \EndOfPage

  % 設定段落之間的距離
  \setlength{\parskip}{0.3cm}

  % Reset page
  \setlength{\parindent}{1.5em}
  \baselineskip=26pt

  % Re-set style
  \pagestyle{plain}
} % End of \def{}

% Summary in Extended Abstract
\newcommand{\ExtAbstractSummary}[1]{
  \begin{framed}
    \ExtAbstractChapter{SUMMARY}{#1}

    % Keyword
    \ExtendedAbstractKeyword
  \end{framed}
} % End of \newcommand{}

% Chapter in Extended Abstract
\newcommand{\ExtAbstractChapter}[2]
{
  \vspace{0.4cm}
  \centerline{\textbf{#1}}

  #2
} % End of \newcommand{}

% Sub-chapter in Extended Abstract
\newcommand{\ExtAbstractSection}[2]
{
  \textbf{#1}

  #2
} % End of \newcommand{}

\newcommand{\AbstractKeyword}[0]
{
  % Keyword
  \par
  \ifthenelse{\equal{\GetKeywordsE}{\empty}}
  {
    \ifthenelse{\equal{\GetKeywordsD}{\empty}}
    {
      \ifthenelse{\equal{\GetKeywordsC}{\empty}}
      {
        \ifthenelse{\equal{\GetKeywordsB}{\empty}}
        {
          \ifthenelse{\equal{\GetKeywordsA}{\empty}}
          {}
          {
            {\noindent \bf 關鍵字:} \GetKeywordsA
          } % End of else{}
        } % End of if{}
        {
          {\noindent \bf 關鍵字:} \GetKeywordsA, \GetKeywordsB
        } % End of else{}
      } % End of if{}
      {
        {\noindent \bf 關鍵字:} \GetKeywordsA, \GetKeywordsB, \GetKeywordsC
      } % End of else{}
    } % End of if{}
    {
      {\noindent \bf 關鍵字:} \GetKeywordsA, \GetKeywordsB, \GetKeywordsC, \GetKeywordsD
    } % End of else{}
  } % End of if{}
  {
    {\noindent\bf 關鍵字:} \GetKeywordsA, \GetKeywordsB, \GetKeywordsC, \GetKeywordsD, \GetKeywordsE
  } % End of else{}
} % End of \newcommand{}

\newcommand{\ExtendedAbstractKeyword}[0]
{
  % Keyword
  \par
  \ifthenelse{\equal{\GetKeywordsE}{\empty}}
  {
    \ifthenelse{\equal{\GetKeywordsD}{\empty}}
    {
      \ifthenelse{\equal{\GetKeywordsC}{\empty}}
      {
        \ifthenelse{\equal{\GetKeywordsB}{\empty}}
        {
          \ifthenelse{\equal{\GetKeywordsA}{\empty}}
          {}
          {
            {\noindent \bf Key words:} \GetKeywordsA
          } % End of else{}
        } % End of if{}
        {
          {\noindent \bf Key words:} \GetKeywordsA, \GetKeywordsB
        } % End of else{}
      } % End of if{}
      {
        {\noindent \bf Key words:} \GetKeywordsA, \GetKeywordsB, \GetKeywordsC
      } % End of else{}
    } % End of if{}
    {
      {\noindent \bf Key words:} \GetKeywordsA, \GetKeywordsB, \GetKeywordsC, \GetKeywordsD
    } % End of else{}
  } % End of if{}
  {
    {\noindent\bf Key words:} \GetKeywordsA, \GetKeywordsB, \GetKeywordsC, \GetKeywordsD, \GetKeywordsE
  } % End of else{}
} % End of \newcommand{}

% ----------------------------------------------------------------------------

