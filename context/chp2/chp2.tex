
\StartChapter{文獻探討}{chapter:文獻探討}


\StartSection{虛擬社群}
本節旨在介紹虛擬社群,首先整理過去學者對虛擬社群的定義,以了解虛擬
社群的意涵與特徵,接著介紹虛擬社群由哪些要素組成,了解這些組成要素對虛
擬社群的重要性,最後介紹虛擬社群的種類,探討不同類型的社群如何滿足不同


\StartSubSection{虛擬社群的定義}
「虛擬社群」是由「實體社群」的概念演化而來的。實體社群是指一群人因
具有共同的興趣或特徵而於現實生活中聚集在一起的組織。相對於實體社群,
「虛
擬」兩字點出虛擬社群的無形性,社群成員不一定會在現實中實際面對面團聚在
一起,但透過電腦科技的輔助,成員們能在網路空間組織社會關係(Lu et al., 2010)。
虛擬社群消弭了時間與距離的限制,使得成員能夠在社群中結交志同道合的朋友,
特別是這些網路上的朋友並不是他們在現實生活中有機會能認識的Wang, Yu,\&Fesenmaier, 2002),虛擬社群提供一個平台讓這些原本不認識的人能夠在網路世
界相遇、相識,在此平台內彼此可透過社會互動產生連結,也藉由此連結關係互
相交換資源(Chiu, Hsu, \& Wang, 2006)。
虛擬社群是個涵蓋多學科的概念,由不同觀點來解釋虛擬社群會有不同的定


義(Gupta \& Kim, 2004; Preece, 2000, 2001),故以下分別由科技、社會學、商業及
綜合多學科的層面來探討虛擬社群的定義。

\begin{description}
\item[由科技的角度來檢視] Lu et al. (2010)認為虛擬社群是以電腦設備及資訊科
技為技術基礎,促使社群成員能在網路空間彼此互相溝通、互動與發展人際關係,
而社群的內容主要由成員共同創作產生。Wasko, Teigland, and Faraj (2009)也指出
虛擬社群是一個使用電腦媒介來創造與維持的溝通系統,具備自我組織、自主、
開放參與的特性。換句話說,當身處任何地點的人們能夠透過以電腦為媒介的通
訊技術經常性地互相溝通時,這群人必然會形成一個社群(Rheingold, 1993)。
\item[以社會學的觀點來探討] Fernback and Thompson (1995)認為虛擬社群是在網
路世界的特定邊界下,透過持續、反覆的互動而形成的社會關係。Lu and Yang
(2011)定義虛擬社群是一種社會網絡關係,成員透過互動來達到分享資訊、知識
與從事社會交往的目的,而社會互動及鑲嵌在社會網絡裡的資源是維持虛擬社群
的重要因素。
\item[由商業角度來探討] ,Dholakia, Bagozzi, and Pearo (2004)認為可將虛擬社群視
為不同大小的消費群體於網路上相識與互動,以達到個人以及社群成員的共同目
標。
\item[由綜合層面來定義] Koh and Kim (2004)將虛擬社群定義為一群具有共同興
趣或目標的人,因存在對知識或資訊的需求而於網路空間互動。Chiu, Hsu, and
Wang (2006)也提出類似的觀點,認為成員不同於一般網路使用者,成員是因為
共同的興趣、目標、需求或習慣而聚集在一起。

\end{description}

由上述多位學者的定義,我們可以歸納出虛擬社群具有以下幾點特徵:
\begin{enumerate}
\item 由一群人組成。
\item 成員因共同的興趣與目標而凝聚在一起。
\item 存在於網路空間。
\item 透過資訊科技互相溝通與互動。
\item 內容是由社群成員協同創作產生。
\item 允許成員在社群中發展社會關係。
\end{enumerate}



\EndChapter