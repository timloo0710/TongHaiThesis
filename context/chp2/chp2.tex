
\StartChapter{文獻探討}{chapter:文獻探討}


\StartSection{虛擬社群}
本節旨在介紹虛擬社群,首先整理過去學者對虛擬社群的定義,以了解虛擬
社群的意涵與特徵,接著介紹虛擬社群由哪些要素組成,了解這些組成要素對虛
擬社群的重要性,最後介紹虛擬社群的種類,探討不同類型的社群如何滿足不同


\StartSubSection{虛擬社群的定義}
「虛擬社群」是由「實體社群」的概念演化而來的。實體社群是指一群人因
具有共同的興趣或特徵而於現實生活中聚集在一起的組織。相對於實體社群,
「虛
擬」兩字點出虛擬社群的無形性,社群成員不一定會在現實中實際面對面團聚在
一起,但透過電腦科技的輔助,成員們能在網路空間組織社會關係(Lu et al., 2010)。
虛擬社群消弭了時間與距離的限制,使得成員能夠在社群中結交志同道合的朋友,
特別是這些網路上的朋友並不是他們在現實生活中有機會能認識的Wang, Yu,\&Fesenmaier, 2002),虛擬社群提供一個平台讓這些原本不認識的人能夠在網路世
界相遇、相識,在此平台內彼此可透過社會互動產生連結,也藉由此連結關係互
相交換資源(Chiu, Hsu, \& Wang, 2006)。
虛擬社群是個涵蓋多學科的概念,由不同觀點來解釋虛擬社群會有不同的定


義(Gupta \& Kim, 2004; Preece, 2000, 2001),故以下分別由科技、社會學、商業及
綜合多學科的層面來探討虛擬社群的定義。

\begin{description}
\item[由科技的角度來檢視] Lu et al. (2010)認為虛擬社群是以電腦設備及資訊科
技為技術基礎,促使社群成員能在網路空間彼此互相溝通、互動與發展人際關係,
而社群的內容主要由成員共同創作產生。Wasko, Teigland, and Faraj (2009)也指出
虛擬社群是一個使用電腦媒介來創造與維持的溝通系統,具備自我組織、自主、
開放參與的特性。換句話說,當身處任何地點的人們能夠透過以電腦為媒介的通
訊技術經常性地互相溝通時,這群人必然會形成一個社群(Rheingold, 1993)。
\item[以社會學的觀點來探討] Fernback and Thompson (1995)認為虛擬社群是在網
路世界的特定邊界下,透過持續、反覆的互動而形成的社會關係。Lu and Yang
(2011)定義虛擬社群是一種社會網絡關係,成員透過互動來達到分享資訊、知識
與從事社會交往的目的,而社會互動及鑲嵌在社會網絡裡的資源是維持虛擬社群
的重要因素。
\item[由商業角度來探討] ,Dholakia, Bagozzi, and Pearo (2004)認為可將虛擬社群視
為不同大小的消費群體於網路上相識與互動,以達到個人以及社群成員的共同目
標。
\item[由綜合層面來定義] Koh and Kim (2004)將虛擬社群定義為一群具有共同興
趣或目標的人,因存在對知識或資訊的需求而於網路空間互動。Chiu, Hsu, and
Wang (2006)也提出類似的觀點,認為成員不同於一般網路使用者,成員是因為
共同的興趣、目標、需求或習慣而聚集在一起。

\end{description}

由上述多位學者的定義,我們可以歸納出虛擬社群具有以下幾點特徵:
\begin{enumerate}
\item 由一群人組成。
\item 成員因共同的興趣與目標而凝聚在一起。
\item 存在於網路空間。
\item 透過資訊科技互相溝通與互動。
\item 內容是由社群成員協同創作產生。
\item 允許成員在社群中發展社會關係。
\end{enumerate}

\StartSection{g0v零時政府}
\StartSubSection{g0v 的緣起:從資訊走向社會}
2012 年 10 月「Yahoo! Open Hack Day」前夕,一組關懷社會議題的資訊人,在看到政府推出的「經濟動能推升方案」廣告後,對其中不對稱、不透明的心態相當不滿,臨時決定更改題目,由原本以電子商務為主題改為中央政府總預算視覺化呈現。在短短三天內,完成了【政府總預算視覺化】專案,是零時政府的第一項成果,也初步凝聚了科技工作者以資訊技術參與公共議題的動能。
2012年12月1日,首次召開「第零次動員戡亂黑客松」,人數超乎預期,社群正式運作。

\StartSubSection{g0v 官網,過去及未來}
g0v.tw 是一個推動資訊透明化的社群,致力於開發公民參與社會的資訊平台與工具。2012 年底開始成形,截至 2015 年初已有 26 場工作坊、受邀演講 30+ 場、媒體報導 100+ 次、500+ 遍佈三大洲的貢獻者,成果皆以自由軟體模式釋出。將 gov 以「零」替代成為 g0v,從零重新思考政府的角色,也是代表數位原生世代從 0 與 1 世界的視野。g0v.tw 以開放原始碼的精神為基底,關心言論自由、資訊開放,寫程式提供公民容易使用的資訊服務。資訊的透明化能幫助公民更確實了解政府運作、更快速了解議題,
不被媒體壟斷,也才可有效監督政府,化為參與行動最終深化民主體質。

\StartSubSection{g0v 的參與者:「沒有人」是萬能的。}
g0v 社群的 motto 是:「不要問為什麼沒有人作這個?先承認你就是『沒有人』,因為『沒有人』是萬能的!」這樣的自主精神吸引了來自四方的參與者,初期以開放源始碼社群為骨幹,包括了網路及軟體業界程式開發、設計的頂尖 hacker 高手、國內外各大資訊公司工程師,陸續加入NGO/NPO工作者、學生、鄉民、新聞/文字工作者、公民創業家、公務人員、資料科學家、藝術視覺影像工作者、大學相關系所教授、各項議題關注者、法律專業人士等。
大家聚在一起,分享自己的專業與心得,一同學習成長,產出成果。


\StartSubSection{g0v 的推動方式:實體線上,開源協作}
g0v.tw 的推動結合線上與實體,以各種網路工具加上兩個月一次的黑客松(註)工作坊,
參與者自主提出專案邀集不同專業者加入,透過 g0v.tw 的實體、線上
平台媒合各方資源,協作出專案成果,不定期會舉辦小型黑客松工作坊。分散式的組織強調自主參與的能量,
鼓勵大家找出問題、提出解決方案、參與討論分享資源,以實作代替按讚。

\StartSubSection{g0v 的發展:跨界多邊交流}
除了以 barcamp/unconference 形式的黑客松活動,社群活動類型也越來越多樣,包括國際年會、教學課程、演講講座、工作坊等。想了解近期 g0v 活動資訊,請前往活動資訊、及演講訊息。從開源社群協作開始,零時政府也逐漸與政府、學界、非政府組織、產業及國際相關機構交流。提出政策建言、合作開設大學課程、實際與 NGO 工作者嘗試新的運動工具與模式,
並串連國際科技公民力量,將成果以及開源協作的文化擴散到各領域。

\StartSection{虛擬社群之知識分享}
本節旨在探討論壇之知識分享,首先介紹論壇中的知識分享情形,瞭解成員在論壇中如何進行知識分享活動,接著介紹知識分享的角色分類,探討論不同角色在知識分享活動上的貢獻程度,
最後介紹論壇知識分享動機,瞭解成員為何願意主動分享知識於論壇中。 


\StartSubSection{論壇中的知識分享情形}
一個虛擬社群若沒有豐富的知識內容,那麼它的價值是有限的(Chiu et al.,
2006; 葉建亨 \& 黃文楨, 2011),因為許多人參與社群活動就是為了尋求知識,不論這些知識的類型為何,皆可以幫助他們解決現實生活中的問題、提升自身的能力、吸收專業知識以及增加創造力(Chen \& Hung, 2010; Chiu et al., 2006; Kim, Song, \& Jones, 2011)。對社群成員來說,虛擬社群提供他們一個與其他人互動的平台,即便原本彼此之間不存在任何社會關係(Bagozzi \& Dholakia, 2002),但透過社會互動成員可以從其他人身上得到資源的機會大為提升(Lu \& Yang, 2011)。現今,虛擬社群被視為是促進知識分享活動的理想工具(Dholakia et al., 2004; Gupta \& Kim, 2004),許多知識皆來自於虛擬社群,特別是專業知識型的社群(Chiu et al., 2006; Lin, Hung, \& Chen, 2009)。
 在眾多知識型的社群中,最常見的型態就是「論壇」(Wasko et al., 2009),其目的在於營造一個資訊交流的環境,吸引對討論主題感興趣的成員於討論區內公開地發表意見、經驗分享及交換資訊(Xu \& Ma, 2006)。一般而言,一個論壇底下有數十個討論區,每個討論區有其特定的討論主題,藉由討論區的分類,可以協助將論壇中的知識分門別類,集中同一討論主題的相關知識於同一討論區內,讓成員可以針對特定主題進行討論。 


\StartSubSection{意見交流區與聊天室}
資料來源:g0v 線上聊天室 (https://g0v-tw.slack.com/messages/general/)
          Hackpad意見交流區(https://g0v.hackpad.com/)
 在意見交流區中,成員的交流方式主要以討論區內的問與答為主,需要知識的成員可在討論區發表文章尋求幫助,
 擁有知識的成員則可以針對感興趣的議題進
行討論或是給予評論(Kim et al., 2011)。在你問我答、一來一往的討論過程中,知識以文字的型態被記錄在論壇中,並藉由圖片、聲音剪輯、影片、網路連結等多媒體的輔助說明,使知識的呈現方式更加多元,同時也能被表達得更為清楚。因此,Lin(2008)認為論壇在知識分享活動上有四個特點:
(1) 建造一個能夠長期保存的成員討論記錄,可作為知識分享的知識庫。
(2) 以文字形式來儲存知識,也可透過多媒體的輔助使知識更完整地保留下來。
(3) 可由不同時區、不同地點的人集結創作。
(4) 藉由主題分類可使知識更有組織地呈現。 
 討論區是論壇成員互動最頻繁的地方,也是知識資源最豐富的地方,因此論壇通常都會提供「搜尋」(Forum Search)功能來協助成員尋找特定主題的文章,讓成員可以在任何時間、不限次數地重複瀏覽過去的討論文章,使得擷取論壇知識的時間較為自由,
 成員也有充分的時間去思考文章內的知識(Guan, Tsai, \& Hwang, 2006),
 讓論壇中的知識分享活動能夠更有效益(Montero, Watts, \& García Carbonell, 2007)。 

\StartSubSection{知識分享的角色分類}
成員互相分享知識是形成社群不可或缺的過程,倘若論壇中沒有知識分享活動,
那麼論壇將無法生存(Butler, 2001; Zhang, Fang, Wei, \& Chen, 2010)。
論壇的知識分享過程中涉及兩個部份,一部份是知識擁有者在論壇中主動貢獻出自己的知識並編纂成文章,使大家能夠透過閱讀了解知識內容;
另一部份則是知識尋求者在論壇中尋找知識並藉由閱讀文章吸收知識,
再去使用知識(Chen \& Hung, 2010)。
一般而言,知識擁有者與知識尋求者這兩種角色並非完全互斥,
同一名論壇成員可能在不同的時間點或不同的知識領域分別扮演這兩種角色。
 除了上述簡單的分類外,Preece (2000)更進一步將論壇使用者分為四種角色,分別為版主、專業評論者、一般參與者與觀望者,分述如下:

\subsubsection{版主(Moderators)} 
主要職責為管理討論區內容,並於成員發生爭執時,以仲裁者的身份排解紛爭,維持討論區的和樂氛圍。一般而言,版主在論壇中扮演監督討論活動的角色,其擁有的論壇權限較一般成員多,若討論區內有不符合討論主題的文章,
版主便可視情況進行刪除、移動等動作,可以把關討論區內的知識資源。

\subsubsection{專業評論者(Professional Commentators)} 
此類型的使用者不一定存在於每個論壇,須視論壇的規則而定,但通常專業評論者為論壇中階級或聲譽較高的會員,代表其擁有較多的專業知識,在知識分享活動中佔有重要地位,可以提供幫助給那些求助的論壇成員,也就是說專業評論者在論壇問與答的活動中,
大多扮演回答問題與提供意見的角色,領導討論活動的進行。

\subsubsection{一般參與者(General Participants)} 
大多數的論壇使用者皆為一般參與者,其在論壇中的主要活動就是在討論區發表文章或是回覆他人文章,
是論壇中知識分享活動的主要參與者。

\subsubsection{觀望者(Lurkers)} 
此類型的使用者並不會主動參與論壇討論活動,
僅沉默地瀏覽討論區內容(Lu \& Yang, 2011)。
在論壇使用者尚未加入論壇會員的階段時,
論壇使用者會以觀望者的身分瀏覽論壇中的知識資源,
並潛伏於論壇中以了解論壇的文化與生態,經過一段時間的探索後,
這些論壇使用者可能會逐漸增加對該論壇的涉入程度,
進而加入該論壇成為會員,主動參與論壇中的活動,並與其他會員建立關係。


\StartSubSection{論壇知識分享動機} 
論壇是以問與答的討論形式為主,知識在你問我答的過程中傳遞出去,達到知識分享的功能,然而經營論壇最大的挑戰是知識的供應,也就是成員知識分享的意願,若成員缺乏知識分享的動機,那他便不會有知識分享的意願,也不會有知識分享的行為產生(Chiu et al., 2006)。
 根據 Kankanhalli, Tan, and Wei (2005)的分類,
 將知識分享動機分為內在因素及外在因素兩種,其內涵解釋如下:




\subsubsection{內在因素} 
內在因素包含樂於助人(Enjoyment in Helping Others)及知識自我效能
(Knowledge Self-efficacy),樂於助人是指成員在沒有預期得到回饋的情況下,主動幫助他人以獲得內心的滿足感,而知識自我效能是指成員對自己有足夠
信心能提供有價值的知識給其他成員,意指當論壇成員認知到他在討論區貢
獻的知識可以幫助到其他人時,他會對自己的專業產生更多的信心,進而提
升知識自我效能;反之,若成員意識到缺乏對其他成員有幫助的知識時,他
就會減少主動參與討論的行為。

\subsubsection{外在因素} 
外在因素包含互惠(Reciprocity)及組織報酬(Organizational Reward),互
角色 知識分享活動是指成員現在之所以願意貢獻知識是因為相信未來當他們需要知識時也能從論壇中得到幫助,
因此,當成員從論壇中取得知識時,他們會覺得有義務回饋論壇(Fang \& Chiu, 2010)。
至於組織報酬則可視為是論壇為鼓勵成員參與討論活動所設置的獎勵制度,
當成員參與越多的討論活動或是貢獻出越有價值的知識時,
他就可以獲得越多的論壇獎勵(Tonteri, Kosonen, Ellonen, \& Tarkiainen, 2011),
常見的獎勵方式如提高帳號權限、提升論壇階級地位、累積聲望值、
獲得虛擬貨幣、於佈告欄公告表揚等等。 
 綜合以上所述,論壇成員的知識分享動機不僅涉及個人認知的內在因素,
 也牽涉到論壇資源等外在因素,因此,論壇經營者除了設立獎勵制度,
 用以鼓勵成員參與知識分享活動外,還可以培養成員樂於助人的論壇文化,
 藉由結合內在與外在因素強化成員知識分享的動機。



\EndChapter