%\chapter{緒論}
\StartChapter{緒論}{chapter:緒論}
隨著科技日新月異,網際網路以及智慧型手機的發明,各式可供下載使用的應用, 不僅讓人們的溝通方式更加便利,能隨時隨地分享新的事物,也改變了人們群聚以及合作的方式,這股強大的虛擬力量也正開始影響媒體以及政府的運作。
網民因為無組織,才能無中生有成為創新組織!——盧希鵬,台灣科技大學管理學院院長






\StartSection{研究背景與動機}

被媒體戲稱「婉君(網軍)」的網路世代,是近期超夯關鍵字。
但你以為,他們只會隱身網路放炮嗎?
其實這群年輕人充分利用網路資源,找到人生方向,且創意、操作手法前所未見,在各領域迸出火花,
震撼了傳統世代。(林士蕙、鄭婷方,2015)

Google台灣區總經理陳俊廷觀察,這群新勢力可用「C世代」形容,因為他們具備4個C的能力。一是具備超高創造內容(Content Creation)的能力,二是隨時黏網(Connect),懂得怎麼善用網路工具暢所欲言,三是懂得社群營造(Community)及分享,四是懂得策展(Curation)的觀念。從每次重大新聞事件一發生,不到幾個小時內,就會有很多網友自製的懶人包、示意圖或影片流出,製作完整程度,還讓主流媒體搶著引用。如今就連傳統世代也難以迴避這股勢力,從去年318學運、一直延燒到22縣市長選舉,野生網路部隊成了影響選舉的關鍵。陳俊廷形容,「台灣人的網路使用思惟,是年輕人引領老人、個人引領企業。」愈年輕,武功愈厲害!——遠見雜誌,第345期


如果你曾在較大的組織中工作,你一定了解管理制度所要付出的代價:時時刻刻都有開不完的會、做不完的文書行政工作、凡事都得經過層層長官批示。這是組織完成任務的必要之惡,不是嗎?但網路顛覆了組織的運作規則,讓管理成本崩盤,因而驅動了自發的集體行為,我們的世界也隨之改變。
「資訊的流通」和「人才的交流」是知識經濟時代中,教育、研究、產業、政府,
乃至於跨領域合作,最為重要的兩件事。歐美各先進國家近年致力於這兩件工作,
以提升其國際競爭力。因此,產官學應規劃如何善用資訊科技、落實資訊基礎建設、
推動資訊教育,來促進資訊的流通;應思考如何突破現有的產業、政府、教育的框架,
摒棄單位只求自身業績表現的本位主義,營造互動、利他的氛圍,
贊助而非以管理者自居,讓如黑客松、Maker Faire、線上課程、
crowdsourcing之類的民間社群自發性的活動,
成為教育、研究、產業、政府發展的重要資源。


本文所要探討的對象--g0v零時政府,是一個網路社群也是個專業平台, 
g0v將gov的o改成0,就是有從零重新思考政府的意味,
也同時代表數位原生代0與1的世界,致力推動資訊透明化、
關心言論自由、資訊開放、從”零”思考政府的角色。並以開源、開放、
開幹的協作模式, 從創辦至今兩年多來促成許多應用政府開放資料的成果。

g0v曾在318太陽花學運(2014年)占領運動中與新聞e論壇、沃草、廣場人並列為新而有力的批判武器。

虛擬社群是一結構極為鬆散的非營利組織,入社的門檻極低,同理,離社的門檻亦極低,
成員之間並沒有很強的約束力,或共同理相,共同願景,這些如散沙般的成員在
g0v裏,尤其特別明顯,其成員有IT人,自由工作者,公務員,學生,社會運動者,上班族,
年紀亦從國/高中生到
中年人都有,研究議題,從社會時髦議題,到冷門的字典文字學,台語文,客家文都有涉入,
這在專案管理上是一個不小的挑戰,
尤其是時間進度的管理,而g0v仍然可以每隔一段時間產出一些具話題性的軟體產品,
促進政府資訊流通,提升政府效能。如一些懶人包的工具。從核能,空污,水資源,公有地的圖文呈現,
促進人民對國家資源更深入的理解。





\StartSection{研究目的}
根據前述的研究動機,本研究目的如下
\begin{enumerate}
\item 如何把分散的成員聚集起來完成一項專案,在過程中知識分享的態度扮演的
   角色。
\item 同上,在過程中,方便記錄討論內容,多人線上編輯及即時匯集討論的
   輔助工具對知識分享的幫助。

\end{enumerate}


\StartSection{研究流程}
根據研究背景及動機,確立研究主題、研究目的與研究範圍,再收集及回顧
國內外相關的研究文獻資料,提出研究架構及確立訪談對象,研擬半結構式訪談
大綱,進行深度訪談後,針對訪談內容以及文獻探討中虛擬社群建構的六大要素,產出問卷調查題目,
再依據研究目的進行研究結果
分析與討論,最後提出本研究的結論與後續研究的建議。  







\StartSection{研究範圍及對象}
本研究是以社群網站g0v零時政府成員為研究對象,
已完成或正在進行中的著名專案為研究範圍。

\StartSection{研究結果分析與討論 }
每一件軟體專案,發起人的背景不盡相同,
有時候是軟體工程師,有時候是社運人士,有些是學生,
有些是老師,有些是語言學家,有時候是關注社會議題,
在HACKPAD上,隨時都有大量的意見,想法產生,
通常要吸引到設計師和工程師加入後,
才慢慢的會有產品的雛型產生,
其中專案規格的逐慚詳盡是一個重點,
規格愈詳細,美工設計師和程式工程師能把任務順利的完成,
其中所舉的兩個專案,在規格的詳細程度,資料的收集上,
都非常具代表性,成員並不一定是該領域的專家,


但願意且樂意知識分享的態度及行為,無償的付出是成功的關鍵。


溝通工具的匯集,如slack, 匯集了多個常用的溝通工具,如IRC,讓訊息可以跨
平台的流通,
而線上同時編輯工具,Hackpad, 讓知識的分享,版本控制,有效而安全。




\StartSection{ 結論與建議}
自由軟體在90年代發展,經過20多年發展,深刻改變了軟體產業的樣貌,
傳統商業公司,如IBM, 微軟,甲骨文,都在自由軟體,開放源始碼的專案中扮
演著重要的角色,而新興公司,如google, facebook, twwiter, 幾乎都是
自由軟體的重度使用公司,創新了很多商業模式,
而手機作業系統Android也是其中一個重要例子。


而自由軟體的風潮開始往不同產業散布,如開源硬體,開源教室…,
開放政府,


而自由軟體的使用及其分享精神,也
大大的影響社會科學,公民政治,
NGO組織,其關鍵仍是知識分享 ,


整個g0v可以研究的地方,
本研究只是一引導讀者入門、做為拋磚引玉之用。






\EndChapter
