\StartChapter{研究設計}{chapter:研究設計}

\StartSection{質性研究取向}
本研究為質性研究取向,乃透過仔細地考察社會現象,不經由統計或量化過程,
來解釋社會生活的一種模式。根據陳向明(2002)對質性研究的定義,認為質性研究
是以研究者本人作為研究工具,在自然情境下,採用多種資料搜集方法,對社會現象
進行整體性描述和探究。質化取向的研究較適於揭露並解釋在表面之下的意義,對於
一些人盡皆知的事也能採取新鮮創意的切入點,可以補充量化方法無法鋪陳的細緻複
雜(徐宗國,1997)。本研究基於以下三方面的考量,開始以採取質性研究方法,
在過程中加入問卷調查產生可量化的研究方法:

\StartSubSection{就研究目的而言}
本研究者期待和受訪者之間的互動,從受訪者個人參與g0v零時政府\&hackpad意見交流區的原因、
經驗和感受,
加以詳細的記錄並分析統整後,歸納出g0v零時政府虛擬社群的歸屬感。
並探討如何經由知識分享再加重強化成員的歸屬感。


\StartSubSection{就研究問題而言}
本研究所探討的問題是針對g0v零時政府社群建構的因素是什麼?
是從虛擬社群會員進行深入描述而分析出建構因素,
而不是事先設定假設有哪些建構因素,
研究者鎖定g0v零時政府一些出色的研究專案進行探討過程中能否發現問題
(從一個社會問題的發想, 蒐集資料到最後形成知識共享)或者提出新的
視角看問題(參與g0v零時政府網站的情感歸屬為何?)。
是故本研究適合個案深入探討,盡量蒐集多來源的資料,確保真實呈現g0v零時政府虛擬社群的特色。

\StartSubSection{就理論發展而言}
本研究採用 Chis, Miriam, Kevin and Ashok(1999)提出虛擬社群六大構成要素為
理論基礎作為問卷調查的題組,期望能在最後歸納出虛擬社群的歸屬感。本研究強調從當
事人的角度,來探討g0v零時政府對論壇看法以及感受。





\StartSection{訪談法之運用:理論與實務}
依據問題之嚴謹程度來進行研究,
我們可以將訪談法細部劃分成四種(Williams,1997;
Minichielloet,1995):

\StartSubSection{結構式訪談(structured interview)}

又稱為標準化訪問,
需事先進行規劃程序
與內容,
題目的數量,預設的答案,答案的用字,造成訪談者的不悅要避免,
通常擇定之訪問問題與答案較為固定,
小心避免淪為考古題的收集,
且訪談過程易於掌控,其優點有利於
統計與驗證理論,只要對其微調變更過的構面進行分析,通常應用於量化研究,

惟其過程缺乏彈性與無隨機應變之可能。

\StartSubSection{無結構式訪談(unstructured interview)}
剛好和前者相反,這種方法的優點是
限制極少、彈性較大,只須規劃一般訪問重點與原則,
字面上看起來,以為是容易的訪談,
通常研究者,最好有上述結構式訪談的經驗,
畢竟受試者可能是此生已接受過十次或上百次
問卷、訪談,
事前準備的不夠完善,
題目描述的不夠清楚,會引起受試者的反彈,常常會有研究中斷的風險。
而缺點就是比結構式訪談費時
無法大規模化;
而且大量的錄音檔,在事後整理上會花費不少時間。
並且訪談過程非標準化難以量化分析。

\StartSubSection{半結構式訪談(semi-structured interview)}
為結構式與非結構式訪談之結合,
開始可對受訪者問些結構性問題,以便蒐集基本資料,
如年紀,從事行業,
經常從事的活動,
居住縣市,
婚姻情況,
然後再問些開放性的問題,
最後是有引導的問題,且題目之前有關連,
一題一題環環相扣,
來
深入瞭解受訪者背景與情境。
\StartSubSection{群體訪談(group interview)}

聚集一群受訪者來共同參與討論研究主題;
最好有多個研究者,
不然會有一些受訪者無事可做的情況,
且較無效率,因為人多意見容易發散,
可採焦點
(focused)或半結構式(semi-structured interview)或深入訪談(in-depth interview)。


\StartSubSection{本研究採取的方法}
本研究在研究設計上採取半結構式的深度訪談法,
利用google表單進行基本資料的分類收集,
了解會眾對分享的認知及行動上的強化及弱化因素。

因為研究者想利用較寬廣的問
題,讓成員缷下心防,且慢慢的依據導引訪談進行,受訪者可以有充足的時間
思考問題,
,而訪談架構只要符合研究問題主要方向,在問題形式不
用太侷限, 附上咖啡,紅茶,讓受訪者有舒適的答題環境,以求認知感受能以趨進真實面貌呈現,
故採用上述訪談法。
其步驟是由
研究者設計的訪談大綱第一次詢問受訪者開放性的問題,可以讓受訪者根據自己經驗、意見
和感受等方面回答後,
研擬共通性的問題,做成電子問卷,收集通識資料,
研究者再適當深入追問(看受試者的反應及意願)和調整問題順序及用字是否明確以作為第三次提問的基礎。




\StartSection{研究、範圍、場域與研究對象及心理、態度 }
g0v.tw 是一個推動資訊透明化的社群,
致力於開發公民參與社會的資訊平台與工具。2012 年底開始成形
。
書面、電子數位文獻: \\
截至 2015 年初已有 26 場工作坊、受邀演講 30+ 場、媒體報導 100+ 次、500+ 遍佈三大洲的貢獻者,成果皆以自由軟體模式釋出。


\InsertImage
[align = center, caption={ 研究流程},
label={fig:chp3_1}]
{./context/chp3/流程圖.png}



\StartSection{虛擬社群 早期組成的重要元素 }
\StartSubSection{提供豐富的內容,且長時間不間斷的一直更新,持續改進}



虛擬社群需要提供會員有興趣或是正需要的資訊,
常常會受限於自己的成立宗旨,技術面向,而讓內容發展有瓶頸,
常常無法持續更新。導致會員流失。
會員因為無新鮮感,無法進行互動討論和搜索所需要的資訊,
讓一個社群緩慢,或快速的走向乏人問津,或關站一途,
早些年的無名小站,及YAHOO 站上好些知名的服務。
豐富且持續更新的資訊內容是吸引網



民最佳利器,也讓會眾增加認同感。如g0v today(http://www.g0v.today/) 內建許多懶人包和一些直播頻道, 
提供會員進行各類社會運動資訊交流,可說是提供相當豐富社會運動相關網站。

針對政府施政的公民關懷,公民參與的零時政府,會眾可以針對政府的政策,社會議題,
及世界開放資料的潮流,
提供源源不絕的構想,改善發展建議及實用App,這點是和其他有目標客群,目標宗旨的社群有極大的不同。


\StartSubSection{提供參與發展、磨練意志力、專長強化的機會}
虛擬社群早期設立的基本宗旨就是網站內容來自於會員集體或個個別參與創作和提供分享,
傳統的版主輪值給予會眾擔任,增加參與論壇未來與眼下的發展,參與愈深入,
也會增進

對社群的向心力。如黑客松網站(http://g0v-jothon.kktix.cc/) 內容資訊大都是由會員張貼和創作, 

採用了去中心化制度提供會員更多不具名參與發展的機會, 
讓有能力的會員貢獻專業能力提升社群社會參與的能力,也讓會員有機會更專業、在職場上更多競爭力。

在過去的三一八學運中,線路組的成員,遇到比過去職場上更嚴苛的例子,
例如一個小區域內,抗議人士同時手機通信,造成基地台失能,斷線的情況,
線路組要將現場影音狀況,傳遞出去,是一個相當嚴苛的不可能任務,
一般職場上不容易遇到的情況。




\StartSubSection{互動、溝通、散布消息的工具}
虛擬社群最歷史悠久的傳播工具,也就是透過「電子佈
告欄」BBS
、
「聊天室」、臉書、Line、...等線上活動,
從早期的純文字,
到現在的有圖有真相,
進一步的影片說明,
提高會眾成員間的互動,
更增進會員間情感(聊天,八卦)和資訊(求職求才、二手3C用品)交
流,對網站的黏著度也會增加。如slack網站(http://
www.) 會眾能夠在論壇討論區做傳統靜態的互動, 群組(GROUP)還有加入好友和私人訊息(簡訊),線上通話的功能, 
並且常舉辦熱門投票區、問卷排行榜、心情分享日記,小秘訣,懶人包... 等線上線下活動, 
就是要讓會員們有更多機會從線上互動轉變為線下走出門戶彼此互動、增進情感。






\StartSubSection{擁有共同的興趣,使命感,道德價值觀,同情心,同理心}
能夠誘發促進會眾(網軍)參與虛擬社群的一些重要關鍵因素就是共同興趣,使命感,道德價值觀,同情心,同理心,
網民早期會因為某些有興

趣的議題而加入網站,但隨著時代的多元化,各種議題雨化春筍般的出現

,
尤其是近期的台灣,經濟議題的嚴重性
淩駕政治性議題,
而政黨的無效能,導致更多人走上街頭,
走上街頭付緒行動的人,
更多是自身的遭遇,如洪仲丘案,如最近的廢死案,
產生了共鳴。
而網路只是提供了便利性,擴大了影響力。






\StartSubSection{拉攏志同道合的成員}
會員採取行動加入社群行動,如每兩個月舉辦的大松,萌典松,
志同道合的好友
也許手頭上有事情在忙,
也許對要進行的專案不是完全的認同,
或是低認同度,
但是成員之間彼些熟識,
也許之前有短暫合作過,
或是對彼此的工作成果有認識,
基本上是認識,廣義的志同道合,
專案領導者,必須再更大膽,
更主動積極的去拉人做事,
不要純粹是聊天,
聊天是一開始的行為,
比聊天進階一點是討論,
意見交流,進而增進
對專案
持續參與不間斷。

積極主動的領導者(專案負責人,召集人)是讓會眾持續參與專案互動關鍵,因為
大多數有才能專長的人,會有某種程度的被動,
需要去挖掘,去拉攏,
如果有非常多
數特質相似或互補的會員協作,就能盡早完成專案。






\StartSubSection{社群交互過程中取得的認同}
黑客松,雖然參加的太多,頻率太高,會有精神體力上的
倦怠感,所以純正的12小時,24小時的馬拉松式的腦力激發,
比較少辦,會是早上9點到下午6點的大松,會場有點遠
離市中心,所以成員不會中途跑掉,綜合以上場景,
一起合作協作的伙伴,可以有長時間的相處,
一些看專案的規格,想需求,貢獻所長,

實幹,苦幹,合作,互助,自然而然形成社群認同,感受越
深,就會更深刻的認同社群,
成員能力的多元化,大家不見得認同的能力選項,只有傳統的
技術選項,還有專案管理選項,美術設計選項,
營運管理選項,愈多愈多的能力付出,可以獲得認同。




\StartSubSection{成員的歸屬感,榮譽感,依賴感}
虛擬社群的歸屬感是建構在長期的共同活動,如出席公民運動,
去協助其他社群架設線路,靠著時間的因素,一點點的累積,
不可能速成,別想要快速達成,緩慢的形成反而關係是緊密,牢固,
當成員產生歸屬感時,
會自發性的將自身個人(小我)的利益和團隊的(大我)權益自然產生緊密連結的效果,
不勉強的自然發生這些情感因素,
也就是會自動自發性無條件
的服務社群,如一些鎖事,而這些鎖事,平常在自己家裏可能不做的瑣事,
如聚會場合的甜點,飲料,披薩,炸雞,炒麵等一些後續處理上會有點
麻煩的飲食,都可見成員,一整天下來,隨時維持環境清潔,
也許在家裏,並不做這些家務事,
但為了相互依賴的情感因素,
,會像自己「家人」一樣付出感情的奉獻。


\StartSubSection{成員的歸屬感,榮譽感,依賴感}


\EndChapter


