\StartChapter{研究設計}{chapter:研究設計}

\StartSection{質性研究取向}
本研究為質性研究取向,乃透過仔細地考察社會現象,不經由統計或量化過程,
來解釋社會生活的一種模式。根據陳向明(2002)對質性研究的定義,認為質性研究
是以研究者本人作為研究工具,在自然情境下,採用多種資料搜集方法,對社會現象
進行整體性描述和探究。質化取向的研究較適於揭露並解釋在表面之下的意義,對於
一些人盡皆知的事也能採取新鮮創意的切入點,可以補充量化方法無法鋪陳的細緻複
雜(徐宗國,1997)。本研究基於以下三方面的考量,而決定採取質性研究方法:

\StartSubSection{就研究目的而言}
本研究者期待和受訪者之間的互動,從受訪者個人參與g0v零時政府\&hackpad意見交流區的原因、經驗和感受,
加以詳細的記錄並分析統整後,歸納出g0v零時政府虛擬社群的歸屬感。是故本研究適合採取自然情境探究而不是實驗情境的探究。
\StartSubSection{就研究問題而言}
本研究所探討的問題是針對g0v零時政府社群建構的因素是什麼?,是從虛擬社群會員進行深入描述而分析出建構因素,而不是事先設定假設有哪些建構因素,研究者鎖定g0v零時政府一些出色的研究專案進行探討過程中能否發現問題(從一個社會問題的發想, 蒐集資料到最後形成知識共享)或者提出新的視角看問題(參與g0v零時政府網站的情感歸屬為何?)。是故本研究適合個案深入探討,盡量蒐集多來源的資料,確保真實呈現g0v零時政府虛擬社群的特色。

\StartSubSection{就理論發展而言}
本研究採用 Chis, Miriam, Kevin and Ashok(1999)提出虛擬社群六大構成要素為
理論基礎作為訪問的題組,期望能在最後歸納出虛擬社群的歸屬感。本研究強調從當
事人的角度,來探討愛越者對論壇看法以及感受。

\StartSection{訪談法之運用}
依據問題之嚴謹度設計,我們可以將訪談法劃分成四種(Williams,1997;
Minichielloet,1995):
\StartSubSection{結構式訪談(structured interview)}
又稱為標準化訪問,需事先進行規劃程序
與內容,通常擇定之訪問問題與答案較為固定,且訪談過程易於掌控,其優點有利於
統計與驗證理論,通常應用於量化研究,惟其過程缺乏彈性與無隨機應變之可能。
\StartSubSection{無結構式訪談(unstructured interview)}
剛好和前者相反,這種方法的優點是
限制極少、彈性較大,只須規劃一般訪問重點與原則,而缺點就是比結構式訪談費時
無法大規模化;並且訪談過程非標準化難以量化分析。
\StartSubSection{半結構式訪談(semi-structured interview)}
為結構式與非結構式訪談之結合,
開始可對受訪者問些結構性問題,以便蒐集基本資料,然後再問些開放性的問題,來
深入瞭解受訪者背景與情境。
\StartSubSection{群體訪談(group interview)}
聚集一群受訪者共同討論研究主題;可採焦點
(focused)或半結構式(semi-structured interview)或深入訪談(in-depth interview)。
本研究在研究設計上採取半結構式的深度訪談法,因為研究者想利用較寬廣的問
題作為訪談依據導引訪談進行,而訪談架構只要符合研究問題,在用字和問題形式不
用太侷限,也希望受訪者認知感受能真實面貌呈現,故採用上述訪談法。其步驟是由
研究者設計的訪談大綱詢問受訪者開放性的問題,可以讓受訪者根據自己經驗、意見
和感受等方面回答後,研究者再適當深入追問和調整問題以作為下一次提問的基礎。




\StartSection{研究場域與研究對象 }
g0v.tw 是一個推動資訊透明化的社群,致力於開發公民參與社會的資訊平台與工具。2012 年底開始成形


\StartSection{虛擬社群 建構的六大要素 }
\StartSubSection{提供豐富的資訊內容(Information)}
虛擬社群需要提供會員有興趣或是正需要的資訊,而這有意義的資訊必須滿足社
群需求,使得會員得以進行互動討論和搜索所需要的資訊,豐富的資訊內容是吸引網
民最佳利器,也讓會員增加歸屬感主要因素之一。如 TOP 寵物網(http://www.topet.net/)
內建構 15 大類討論區和 28 個子討論區,提供會員進行各類寵物照顧醫療資訊交流,
並在某些討論區上提供駐站專家進行問題回答,還可以上搜尋想要知道的寵物相關資
訊,可說是提供相當豐富寵物相關網站。
\StartSubSection{提供參與發展的機會(Development)}
虛擬社群設立的基本宗旨就是網站內容資訊形成是來自於會員參與創作和提供,
更是開放討論區的版主給予會員擔任增加參與發展程度,參與發展機會越多也會增進
對網站的歸屬感。如早安越南網站(http://www.abubook.com/)內容資訊大都是由會
員張貼和創作,採用了版主制度提供會員更多參與發展的機會,讓有能力的會員貢獻
專業能力提升網站專業程度,也讓會員有更專業、更豐富資訊來源。
\StartSubSection{彼此互動(Interaction)}
虛擬社群形成中要發展與其他使用者做有意義互動的機會,也就是透過「電子佈
告欄」
、
「聊天室」...等線上活動,提高會員成員間的互動更增進會員間情感和資訊交
流,對網站的歸屬感也會增加。如前進越南網站(http://www.seeviet.net/bbs/forum.php/)
會員能夠在論壇討論區互動,還有加入好友和私人訊息的功能,並且常舉辦熱門投票
區、排行榜、心情分享日誌...等線上活動,就是要讓會員們有更多機會彼此互動、增
進情感,進而增加對網站的歸屬感。
\StartSubSection{擁有共同的興趣(Common interest)}
能夠誘發人們參與虛擬社群的潛在因素之一就是共同興趣,網民會因為某些有興
趣的議題而加入網站,就如同實體社群中俱樂部一般,有了共同興趣才能發展出加入
會員的開始,而對這共同興趣越強烈相對的對網站越有歸屬的感覺。如愛拍照人像攝
14影網站(http://www.i-photo.com.tw/test/frameset.htm)就是一個共同興趣非常顯明的攝
影網站,他們不是一群專業的攝影家,而是一群喜愛拍攝人像的業餘拍攝人士,長期
舉辦戶外、棚內人像拍攝,藉由網站彼此互動學習和號召有共同興趣者加入,有如實
體的攝影社團一般。
\StartSubSection{與成員有志同道合之感(Like-minded)}
會員加入網站後是否有志同道合的好友是持續參與網站互動關鍵,因為如果有多
數而且特質相似的會員認識,就會在特別主題下進行討論、意見交流,進而增進對網
站持續參與不間斷。如神來也遊戲網(http://www.godgame.com.tw/)中共有 19 種不
同遊戲區域,而每一個遊戲裡會員都可以由志同道合的好友成立公會進行公會之間戰
爭,因為有公會之間的戰爭,也增加了許多會員持續參與網站的動機。
\StartSubSection{社群認同(Commitment)}
會員對於虛擬社群和站長所形成的親和力是否能夠感受到就是社群認同,感受越
深社群認同程度就越深,會員就較高意願分享資訊和資源,就更能相互扶持對社群共
同信念目標努力。如:史萊姆第一個家(http://www.slime.com.tw/)就是一個免費資
源分享虛擬社群,裡面每天都有會員免費上傳有用的程式和遊戲,還有使用後討論心
得感想和使用介紹,最重要的是這些資源和諮詢都是免費的,站長更是用心的回答和
將眾多資訊分類以方便使用,是一個成功的社群認同度極高的免費資源分享網站。
\StartSubSection{歸屬感(Sense of belonging)}
虛擬社群的歸屬感是建構在六大虛擬社群之上,當團體中的成員產生歸屬感時,
會自發性的將自身的利益和團體的權益產生連結的效果,也就是會自發性無附加條件
的服務社群,就像服務自己「家人」一樣無私的奉獻。愛越者虛擬社群是否能有如此
的歸屬感呢?若有!那種歸屬感是什麼呢?若沒有!那又是什麼原因呢?待本研究
使用六大建構要素建構出愛越者虛擬社群的歸屬感。




\EndChapter


