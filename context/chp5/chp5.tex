\chapter{結論與建議}
自由軟體、開放源代碼運動,一開始只是MIT菁英黑客們在學生時期,
和志同道合的伙伴形成一個技術上的烏托邦共享次文化。

三十幾年前,這些學生實驗了,沒有階級,沒有過度的智慧財產保護,很純粹的技術交流及分享,
大家的知識成果無阻礙的分享、修改、再分享,只是讓主機運作的更良好,鑽研如何讓程式寫的
更短更有效率。

然後慢慢的從圈子裏流傳出來,在和g0v訪談成員中,有次談到,即使是非營利組織,
取得公部門資料,並不見得願意、或有意識到應該分享出來,所以分享、互利互惠的
觀念及行動,在軟體工程圈成熟並行之有年,但開放政府,開放資料,公民參與的領域,
還有進步的空間。



\section{研究結論}
做為這跨界領域的先行者g0v,總體而言,表現得可圈可點。
一些執行上的細節可以微調,

在軟體工具上,如Hackpad,
有些可改進的地方:
\begin{itemize}
\item 斷線就不能用
\item 一定要連線 搜尋結果不一定有想找的東西 中文打字有問題
\item 視覺呈現有點混亂 第一次使用不習慣
\item 除了視覺化的排版,或許可加入 markdown 的支援,讓不同文件格式可以有機會匯入
\item 不能輸出為文字檔
\item 有點lag(延遲)
\end{itemize}

而活動的安排上,
有些可改進的地方:
\begin{itemize}
\item 技術面更加親民,才能吸引更多一般民眾
\item 可能需要更有效的組織
\item 可以想想在「學習」這一塊,除了一些偶而會開的課。如何讓這些課與進行中的專案結合
\item 新手入門指引
\item 對不會寫程式的人,第一次參加,會無法很快進入狀況
\item 討論和申請介面還是太複雜,不曉得有沒有可能變成類似104工作。\\ 大家可以在上頭申請媒和。
\item 有些專案需要有更多與政府互動的機會
\item 希望 g0v 能夠繼續保持下去. 莫忘初衷.

\end{itemize}

本論文研究就以 【希望 g0v 能夠繼續保持下去. 莫忘初衷】做為結束。


\section{研究限制}

深入了解後,發現參與活動(大,小松)的成員,新進人員居多(近半數),
所以在活動現場,當面訪談的對象,比較有空的人,多是初次參加,
手頭上有專案在進行,主辦或協辦的多是資深成員,這些人是主要訪談標的,
但通常較忙,這是在進行接觸訪談時須注意的地方,
資深成員可透過線上聊天室(slack)進行,
有些資深成員,並不喜歡用線上聊天室(slack),必須在活動現場當面跟他預約上線訪談的時段,
以免等了好幾個晚上,受訪者未上線的窘境。

一般網路社群,因為資訊能力較高,對個資的議題較敏感、重視,所以手機號,email帳號,一般
聯絡方式都不喜歡交換,重視個人隱私,往來訊息,以聊天室,共筆區,臉書,郵件論壇為主。
這些是訪談須注意的地方。

研究者最好多出席活動,實際做出貢獻,講求實作的社團,有貢獻(文字,影像,程式碼)
會比較容易得到認同,有了認同,成員會比較樂意後續的配合。研究者需要時間去參與。

全面性,廣泛的進行問卷,較難進行,因為
動民主  \footnotemark[1] 的關係。要動員分散的成員一直來填問卷,
並不容易,還是須配逐一徵詢受測意願,少數成員會看到公告就自行主動填問卷。

\footnotetext[1]{社群決策機制:動民主

g0v社群分散作業,鮮少有需要共同決策的事項,
但如果需要討論則採用【動民主】(liquid feedback)系統,進行討論審議投票。
成為g0v 貢獻者即有審議/投票權。
(參與專案、並由專案發起人推薦即可成為貢獻者。)由此登入【動民主】。
}

實際參與社群活動時間須以一年以上為合適,本研究參與了七個月左右,對社群的認識仍不充分。


\section{後續研究建議}
本次研究的主題主要以成員為主,
其實g0v有很多極富特色的專案,
專案中涉及很多公民議題,任選一項都足以開展成研究。
附錄中有一些代表性專案參考。
