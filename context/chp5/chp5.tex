\chapter{結論與建議}
自由軟體、開放源代碼運動,一開始只是MIT菁英黑客們在學生時期,
和志同道合的伙伴形成一個技術上的烏托邦共享次文化。

三十幾年前,這些學生實驗了,沒有階級,沒有過度的智慧財產保護,很純粹的技術交流及分享,
大家的知識成果無阻礙的分享、修改、再分享,只是讓主機運作的更良好,鑽研如何讓程式寫的
更短更有效率。

然後慢慢的從圈子裏流傳出來,在和g0v訪談成員中,有次談到,即使是非營利組織,
取得公部門資料,並不見得願意、或有意識到應該分享出來,所以分享、互利互惠的
觀念及行動,在軟體工程圈成熟並行之有年,但開放政府,開放資料,公民參與的領域,
還有進步的空間。

做為這跨界領域的先行者g0v,總體而言,表現得可圈可點。
一些執行上的細節可以微調,

在軟體工具上,如Hackpad,
有些可改進的地方:
\begin{itemize}
\item 斷線就不能用
\item 一定要連線 搜尋結果不一定有想找的東西 中文打字有問題
\item 視覺呈現有點混亂 第一次使用不習慣
\item 除了視覺化的排版,或許可加入 markdown 的支援,讓不同文件格式可以有機會匯入
\item 不能輸出為文字檔
\item 有點lag(延遲)
\end{itemize}

而活動的安排上,
有些可改進的地方:
\begin{itemize}
\item 技術面更加親民,才能吸引更多一般民眾
\item 可能需要更有效的組織
\item 可以想想在「學習」這一塊,除了一些偶而會開的課。如何讓這些課與進行中的專案結合
\item 新手入門指引
\item 對不會寫程式的人,第一次參加,會無法很快進入狀況
\item 討論和申請介面還是太複雜,不曉得有沒有可能變成類似104工作。\\ 大家可以在上頭申請媒和。
\item 有些專案需要有更多與政府互動的機會
\item 希望 g0v 能夠繼續保持下去. 莫忘初衷.

\end{itemize}

本論文研究就以 【希望 g0v 能夠繼續保持下去. 莫忘初衷】做為結束。


\section{研究結論}

\section{研究限制}
\section{後續研究建議}
