\StartChapter{實証分析}{chapter:實証分析}
筆者於今年一月開始,陸續每月從台中北上參與g0v位在台北的聚會活動,
基本上做為以實幹為主的社團,有勇氣從家裏走出來的成員,對於筆者
所提出來的訪談需求,通常是樂於回答的。

在訪談過程中,可以了解政府的言論及政策會影響g0v的運作方式,
從最早期,參與台灣雅虎舉辦的資訊比賽,
以政府預算的數字資料視覺化,可看出創社成員的公民關心政治導向,
資訊工具技能不是只做為玩樂、炫技或為生活便利服務,
而解放放大了黑客技能於社會關懷,
以溫和而理性的方式,有時幽默,有時諧謔,冷靜的用數字視覺化呈現事實的真相。

自由軟體,開放源始碼運動在台灣行之近二十年,
但其參與成員,基本上,沒有太大的改變,以電腦或3C產品的重度使用者為主,
以軟體工程師為大宗,大約佔9成,
少部分的法律專家(專利、著作權相關),理工科學生、學者,中、小學老師,
組成較單一的成員,台灣的資訊從業人口總數不多,且新軟體工程技術又層出不窮的情況下,
導致以資訊工程師為主的社群,常常呈現曇花一現的狀況,有些辦一次活動,社群就結束了,
有些撐半年,一般在兩年內,工程師社群就會從熱絡不絕變成門可羅雀,
而g0v在2012年成立,至今已突破兩年,人氣依舊持續不降,顯然是個值得探究的現象。
經過訪談的過程中,總結了一些g0v特色。

\begin{enumerate}
\item 專案本土化,不再以追隨時髦流行技術為主,以新技術為輔。
\item 專案方向加入了社會關懷、公民參與,吸引了非資訊背景成員加入。
\item 三一八太陽花學運的參與,吸引了更多的設計師加入。
\item 設計師的參與人數較一般軟體社群為多,且活躍。
\item 社運人士,或NGO,都會資訊化,數據視覺化的需求,會優先來g0v找同伴。
\item 公民戶外運動,從原來的硬體、線路工程師技術提升。吸引一些獨立媒體社群關注。
\item 維持自主自發性參與,不設定明確目標,明確完成時限來推動專案
\item 不成立公司,維持目前社群的鬆散架構,樂捐金額可供支出活動的 \\
       飲食支出即可,讓財務簡單透明。
\item 部分成員到公部門、學校、社團演講,持續宣揚理念
\item 和公民組織,非營利組織合辦活動,及媒體關係良好。
\item 知識分享的行為持續不間斷,所有g0v的知識產出授權都是創用CC
\end{enumerate}

在訪談過程中,初步有了問卷題目的方向,經由兩次的測試(含用字遣詞的修正),才進行正式問卷調查。





\StartSection{統計結果輸出}
\StartSubSection{成員一般調查,年紀、性別、職業}


\InsertImage
[align = center,scale=0.75, caption={成員年齡},
label={fig:stat01}]
{./context/chp4/01_age.PNG}

\begin{center} 
\begin{table}[h]
\centering
\caption{成員年齡}
\label{4_1_age}
\begin{tabular}{ccc} \hline
年齡&	人數 &	佔比\% \\ \hline
10-20&	0 &	0\% \\ \hline
21-30&	22&	43.1\% \\ \hline
31-45&	27&	52.9\% \\ \hline
46-50&	2&	3.9\% \\ \hline
51-60&	0&	0\% \\ \hline
60以上&	0&	0\% \\ \hline

\end{tabular}
\end{table}
\end{center}



\InsertImage
[align = center,scale=0.75, caption={成員性別},
label={fig:stat02}]
{./context/chp4/02_sex.PNG}

\begin{center} 
\begin{table}[h]
\centering
\caption{成員性別}
\label{4_1_sex}
\begin{tabular}{ccc} \hline
性別&	人數 &	佔比\% \\ \hline
女&	14&	27.5\% \\ \hline
男&	37&	72.5\% \\ \hline

\end{tabular}
\end{table}
\end{center}




\InsertImage
[align = center,scale=0.75, caption={成員職業分布},
label={fig:stat03}]
{./context/chp4/03_prof.PNG}

\begin{center} 
\begin{table}[h]
\centering
\caption{職業分布}
\label{4_1_prof}
\begin{tabular}{ccc} \hline
職業別&	人數 &	佔比\% \\ \hline
自由工作者&	8&	15.7\% \\ \hline
資訊業&	19&	37.3\% \\ \hline
製造業&	2	&3.9\% \\ \hline
非營利組織,社福團體&	5&	9.8\% \\ \hline
公民組織&	1&	2\% \\ \hline
文字工作者&	2&	3.9\% \\ \hline
設計師	&6&	11.8\% \\ \hline
學生	&4&	7.8\% \\ \hline
金融業	&0&	0\% \\ \hline
其他	&14&	27.5\% \\ \hline
\end{tabular}
\end{table}
\end{center}





\StartSubSection{身體健康對分享的調查}

Q:健康的身體有加強您分享的意願嗎?
\InsertImage
[align = center,scale=0.75, caption={成員意見:身體健康促進分享 },
label={fig:stat04}]
{./context/chp4/12_health_force.PNG}

\begin{center} 
\begin{table}[h]
\centering
\caption{身體健康促進分享}
\label{4_1_health_force}
\begin{tabular}{ccc} \hline
身體健康是否促進分享&	人數 &	佔比\% \\ \hline
有加強&	37	&72.5\% \\ \hline
沒加強&	7	&13.7\% \\ \hline
\end{tabular}
\end{table}
\end{center}



Q:分享會不會有益您的身心健康?
\InsertImage
[align = center,scale=0.75, caption={成員意見:分享促進身心健康},
label={fig:stat05}]
{./context/chp4/10_health.PNG}


\begin{center} 
\begin{table}[h]
\centering
\caption{分享促進身心健康}
\label{4_1_health}
\begin{tabular}{ccc} \hline
是否有益身心&	人數 &	佔比\% \\ \hline
不會:1&	2&	3.9\% \\ \hline
2	&1	&2\% \\ \hline
3	&6	&11.8\% \\ \hline
4	&19	&37.3\% \\ \hline
會:5&	23&	45.1\% \\ \hline
\end{tabular}
\end{table}
\end{center}








\StartSubSection{分享的一些細節}


Q:參與活動(小松,大松),您會花多少時間準備呢?
\InsertImage
[align = center,scale=0.75, caption={分享的因素分析:時間長度},
label={fig:stat06}]
{./context/chp4/15_time.PNG}

\begin{center} 
\begin{table}[h]
\centering
\caption{行前準備時間}
\label{4_1_time}
\begin{tabular}{ccc} \hline
行前準備時間&	人數 &	佔比\% \\ \hline
<30分鐘	&11	&21.6\% \\ \hline
30-60	&9	&17.6\% \\ \hline
60~120	&9	&17.6\% \\ \hline
>120	&5	&9.8\% \\ \hline
放空,沒準備	&14&	27.5\% \\ \hline
其他	&3	&5.9%
\end{tabular}
\end{table}
\end{center}





Q:每月,您會安排幾日用來參與活動呢?
\InsertImage
[align = center,scale=0.75, caption={分享的因素分析:每月幾天},
label={fig:stat07}]
{./context/chp4/16_day.PNG}

\begin{center} 
\begin{table}[h]
\centering
\caption{參與活動天數}
\label{4_1_day}
\begin{tabular}{ccc} \hline
參與活動天數&	人數 &	佔比\% \\ \hline
<=2	&27	&54\% \\ \hline
3-5	&20	&40\% \\ \hline
6-10&	1&	2\% \\ \hline
>=11&	2&	4\% \\ \hline 
\end{tabular}
\end{table}
\end{center}




Q:分享時,您會分隨意之作、精心創作,而用不同的授權嗎?
\InsertImage
[align = center,scale=0.75, caption={分享之偏心、無偏心},
label={fig:stat10}]
{./context/chp4/27_bias.PNG}


\begin{center} 
\begin{table}[h]
\centering
\caption{不同的授權}
\label{4_1_bias}
\begin{tabular}{ccc} \hline
不同的授權&	人數 &	佔比\% \\ \hline
不會:1	&15	&29.4\% \\ \hline
2	&8&	15.7\% \\ \hline
3	&10	&19.6\% \\ \hline
4	&13	&25.5\% \\ \hline
會:5&	5&	9.8\% \\ \hline
\end{tabular}
\end{table}
\end{center}


Q:分享內容,您會刻意把內容更淺白的表達,懶人包式的表達?
\InsertImage
[align = center,scale=0.75, caption={分享之品質:懶人包},
label={fig:stat10}]
{./context/chp4/28_content_clear.PNG}

\begin{center} 
\begin{table}[h]
\centering
\caption{內容深入淺出}
\label{4_1_content_clear}
\begin{tabular}{ccc} \hline
淺白嗎&	人數 &	佔比\% \\ \hline
不會:1&	2&	3.9\% \\ \hline
2	&3	&5.9\% \\ \hline
3	&7	&13.7\% \\ \hline
4	&27	&52.9\% \\ \hline
會:5&	12&	23.5\% \\ \hline
\end{tabular}
\end{table}
\end{center}




Q:您在發布分享文件時,會看過幾次才發布?
\InsertImage
[align = center,scale=0.75, caption={分享品質},
label={fig:stat10}]
{./context/chp4/24_times.PNG}


\begin{center} 
\begin{table}[h]
\centering
\caption{分享品質}
\label{4_1_times}
\begin{tabular}{ccc} \hline
次數&	人數 &	佔比\% \\ \hline
1次&	11&	21.6\% \\ \hline
2次&	7&	13.7\% \\ \hline
3次&	3&	5.9\% \\ \hline
4次&	0&	0\% \\ \hline
5次&	1&	2\% \\ \hline
0次&	22&	43.1\% \\ \hline
\end{tabular}
\end{table}
\end{center}




Q:關於"分享"這種能力,您覺得不常鍛鍊,會鈍化,落伍跟不上嗎?
\InsertImage
[align = center,scale=0.75, caption={分享要持續},
label={fig:stat10}]
{./context/chp4/34_offen_share.PNG}

\begin{center} 
\begin{table}[h]
\centering
\caption{分享要持續精進}
\label{4_1_offen}
\begin{tabular}{ccc} \hline
分享力鈍化&	人數 &	佔比\% \\ \hline
不會:1&	4&	7.8\% \\ \hline
2	&8&	15.7\% \\ \hline
3	&13&	25.5\% \\ \hline
4	&14	&27.5\% \\ \hline
會:5&	12	&23.5\% \\ \hline

\end{tabular}
\end{table}
\end{center}

\StartSubSection{制度的設計是否影響分享}

Q:g0v推行的參與者(貢獻者)的技能貼紙,您認為有助於專案的進行嗎?
\InsertImage
[align = center,scale=0.75, caption={分享的因素分析:人才分類},
label={fig:stat08}]
{./context/chp4/20_classify.PNG}


\begin{center} 
\begin{table}[h]
\centering
\caption{技能貼紙}
\label{4_1_classify}
\begin{tabular}{ccc} \hline
是否有效&	人數 &	佔比\% \\ \hline
不會:1&	0&	0\% \\ \hline
2	&1	&2\% \\ \hline
3	&12&	23.5\% \\ \hline
4	&16&	31.4\% \\ \hline
會:5&	22&	43.1\% \\ \hline
\end{tabular}
\end{table}
\end{center}



Q:職能貼紙方案做的好,您認為有加強您分享的力道嗎?
\InsertImage
[align = center,scale=0.75, caption={分享的因素分析:人才分類效果},
label={fig:stat09}]
{./context/chp4/21_classify_force.PNG}

\begin{center} 
\begin{table}[h]
\centering
\caption{人才分類有效性}
\label{4_1_classify_force}
\begin{tabular}{ccc} \hline
是否有效&	人數 &	佔比\% \\ \hline
不會:1&	1	2\% \\ \hline
2	&3	&5.9\% \\ \hline
3	&16&	31.4\% \\ \hline
4	&14	&27.5\% \\ \hline
會:5&	17	&33.3\% \\ \hline

\end{tabular}
\end{table}
\end{center}



\StartSubSection{營造競爭的氛圍}

Q:您(成員)認為分享者之間,會互相競爭嗎?
\InsertImage
[align = center,scale=0.75, caption={分享品質之維持競爭力},
label={fig:stat10}]
{./context/chp4/25_competive.PNG}

\begin{center} 
\begin{table}[h]
\centering
\caption{互相競爭}
\label{4_1_competive}
\begin{tabular}{ccc} \hline
是否互相競爭&	人數 &	佔比\% \\ \hline
沒有:1&	11&	21.6\% \\ \hline
2	&12	&23.5\% \\ \hline
3	&11	&21.6\% \\ \hline
4	&14	&27.5\% \\ \hline
有:5	&3&	5.9\% \\ \hline

\end{tabular}
\end{table}
\end{center}



Q:分享的內容提高那些方面,有助於脫穎而出?
\InsertImage
[align = center,scale=0.75, caption={分享之多方面競爭},
label={fig:stat10}]
{./context/chp4/26_compete_aspect.PNG}

\begin{center} 
\begin{table}[h]
\centering
\caption{內容取向}
\label{4_1_aspect}
\begin{tabular}{ccc} \hline
內容取向&	人數 &	佔比\% \\ \hline
吸引目光的競爭(圖文重口味)&	13&	31.7\% \\ \hline
標題要很炫,或是有點難&	10&	24.4\% \\ \hline
內容品質&	33&	80.5\% \\ \hline
團隊整合力&	15&	36.6\% \\ \hline
媒體資源&	4&	9.8\% \\ \hline
政府資源&	3&	7.3\% \\ \hline
其他	&2&	4.9\% \\ \hline
\end{tabular}
\end{table}
\end{center}


\StartSubSection{多數人的分享及回饋}

Q:那些狀況之下,您會對分享心灰意冷?
\InsertImage
[align = center,scale=0.75, caption={影響分享},
label={fig:stat10}]
{./context/chp4/32_frustuctive.PNG}

\begin{center} 
\begin{table}[h]
\centering
\caption{分享的情緒}
\label{4_1_frust}
\begin{tabular}{ccc} \hline
狀況&	人數 &	佔比\% \\ \hline
專案沒人理	&40	&78.4\% \\ \hline
友人不認同	&7	&13.7\% \\ \hline
家人不看好	&4	&7.8\% \\ \hline
其他	9	&17.6\% \\ \hline
\end{tabular}
\end{table}
\end{center}



您平常(最近1個月)有往來的g0v成員,大約有幾人?
\InsertImage
[align = center,scale=0.75, caption={分享要大家一起來},
label={fig:stat10}]
{./context/chp4/39_withothers.PNG}

\begin{center} 
\begin{table}[h]
\centering
\caption{一起分享}
\label{4_1_withothers}
\begin{tabular}{ccc} \hline
合作人數&	人數 &	佔比\% \\ \hline
1-3人	&27&	56.3\% \\ \hline
4-6人	&12&	25\% \\ \hline
7-10人	&7&	14.6\% \\ \hline
11-20人	&1&	2.1\% \\ \hline
>20人	&1&	2.1\% \\ \hline

\end{tabular}
\end{table}
\end{center}


\StartSubSection{其他}
Q:離洪仲丘,太陽花這些大事件一段時間,那時的技術,會不會覺得最近無用武之地?
\InsertImage
[align = center,scale=0.75, caption={其他},
label={fig:stat10}]
{./context/chp4/33_nouse.PNG}

\begin{center} 
\begin{table}[h]
\centering
\caption{其他}
\label{4_1_other}
\begin{tabular}{ccc} \hline
不會(1)..\\ 會(5)&	人數 &	佔比\% \\ \hline
不會:1&	21&	41.2\% \\ \hline
2	&22	&43.1\% \\ \hline
3	&6	&11.8\% \\ \hline
4	&1	&2\% \\ \hline
會:5&	1&	2\% \\ \hline

\end{tabular}
\end{table}
\end{center}


\StartSection{結果分析}

由研究結果可知,


\StartSubSection{分享與健康的關係}
分享者經由分享促進身體健康,
而健康也讓他們更樂於分享。互相增強的效果。

\StartSubSection{分享者與外出參與外動}
每個月至少花兩天,至多十天以上,來參與活動,
不只宅在家中電腦前。
而每次參與活動,都會花時間,少則半小時,多則一、二小時準備活動相關的事項。

\StartSubSection{分享者體貼的關心讀者與使用者}
過半數的分享者會把用淺顯的方式表達內容,
而在發布內容時,大部分的人會快速分享,
符合開放源始碼界的一個悠久傳統,
儘早發布,儘早修正錯誤。
對於精心發布的內容,會用不同的創用CC授權,
所以分享者會去了解創用CC授權,各種授權的差異,適當的維護自己的勞動成果,
而分享者除了保持常常分享的頻率及習慣外,
也要與時俱進的學習一些流行的傳播工具,以免自己的分享力鈍化了。

\StartSubSection{讓制度傾聽分享者心聲}
分享者並非全能,
有設計師、前端工程師、產品經理、社運人士、環保人士…等,
g0v成員提出了職能貼紙的制度,
讓成員可以把貼紙名牌吊在身上,讓職能視覺後,
找到需要的人才,快速進行專案。
分享者大多認為這是有效的制度。

\StartSubSection{分享者之間}
受訪者之間普遍認為沒有互相競爭的情況,
不過,因為專案需要不同的人才進入,
而人才是有限的,所以專案的各個層面,必須加強內容,以吸引人才加入。

\StartSubSection{分享後的回饋}
去關注分享者提出的內容,常回應,會讓分享者快樂,他是非常害怕乏人問津的情況 。
分享者之間會密切的往來,少則與一兩人進行討論,多則與十多人保持合作關係。


傳統工程師社群,年輕男性居多,職業以資訊業為主,
由上一節的研究,可以了解到,
成員的多元組成,例如:男女比例不要太過懸殊,中年人也多多參與,
各行各業的人都能加入,
,除了可以活絡社群的互動氣氛,
延長社群的平均壽命,提升分享的樂趣。




\EndChapter
