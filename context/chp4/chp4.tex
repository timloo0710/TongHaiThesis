\StartChapter{實証分析}{chapter:實証分析}
筆者於今年一月開始,陸續每月從台中北上參與g0v位在台北的聚會活動,
基本上做為以實幹為主的社團,有勇氣從家裏走出來的成員,對於筆者
所提出來的訪談需求,通常是樂於回答的。

在訪談過程中,可以了解政府的言論及政策會影響g0v的運作方式,
從最早期,參與台灣雅虎舉辦的資訊比賽,
以政府預算的數字資料視覺化,可看出創社成員的公民關心政治導向,
資訊工具技能不是只做為玩樂、炫技或為生活便利服務,
而解放放大了黑客技能於社會關懷,
以溫和而理性的方式,有時幽默,有時諧謔,冷靜的用數字視覺化呈現事實的真相。

自由軟體,開放源始碼運動在台灣行之近二十年,
但其參與成員,基本上,沒有太大的改變,以電腦或3C產品的重度使用者為主,
以軟體工程師為大宗,大約佔9成,
少部分的法律專家(專利、著作權相關),理工科學生、學者,中、小學老師,
組成較單一的成員,台灣的資訊從業人口總數不多,且新軟體工程技術又層出不窮的情況下,
導致以資訊工程師為主的社群,常常呈現曇花一現的狀況,有些辦一次活動,社群就結束了,
有些撐半年,一般在兩年內,工程師社群就會從熱絡不絕變成門可羅雀,
而g0v在2012年成立,至今已突破兩年,人氣依舊持續不降,顯然是個值得探究的現象。
經過訪談的過程中,總結了一些g0v特色。

\begin{enumerate}
\item 專案本土化,不再以追隨時髦流行技術為主,以新技術為輔。
\item 專案方向加入了社會關懷、公民參與,吸引了非資訊背景成員加入。
\item 三一八太陽花學運的參與,吸引了更多的設計師加入。
\item 設計師的參與人數較一般軟體社群為多,且活躍。
\item 社運人士,或NGO,都會資訊化,數據視覺化的需求,會優先來g0v找同伴。
\item 公民戶外運動,從原來的硬體、線路工程師技術提升。吸引一些獨立媒體社群關注。
\item 維持自主自發性參與,不設定明確目標,明確完成時限來推動專案
\item 不成立公司,維持目前社群的鬆散架構,樂捐金額可供支出活動的 \\
       飲食支出即可,讓財務簡單透明。
\item 部分成員到公部門、學校、社團演講,持續宣揚理念
\item 和公民組織,非營利組織合辦活動,及媒體關係良好。
\item 知識分享的行為持續不間斷,所有g0v的知識產出授權都是創用CC
\end{enumerate}

在訪談過程中,初步有了問卷題目的方向,經由兩次的測試,進行問卷調查。





\StartSection{統計結果輸出}



\StartSection{結果分析}

\InsertImage
[align = center,scale=0.5, caption={圖 4-1 統計結果},
label={fig:stat01}]
{./context/chp4/統計01.png}

\InsertImage
[align = center,scale=0.5, caption={圖 4-2 統計結果},
label={fig:stat02}]
{./context/chp4/統計02.png}

\InsertImage
[align = center,scale=0.5, caption={圖 4-3 統計結果},
label={fig:stat03}]
{./context/chp4/統計03.png}

\InsertImage
[align = center,scale=0.5, caption={圖 4-4 統計結果},
label={fig:stat04}]
{./context/chp4/統計04.png}

\InsertImage
[align = center,scale=0.5, caption={圖 4-5 統計結果},
label={fig:stat05}]
{./context/chp4/統計05.png}

\InsertImage
[align = center,scale=0.5, caption={圖 4-6 統計結果},
label={fig:stat06}]
{./context/chp4/統計06.png}

\InsertImage
[align = center,scale=0.5, caption={圖 4-7 統計結果},
label={fig:stat07}]
{./context/chp4/統計07.png}

\InsertImage
[align = center,scale=0.5, caption={圖 4-8 統計結果},
label={fig:stat08}]
{./context/chp4/統計08.png}

\InsertImage
[align = center,scale=0.5, caption={圖 4-9 統計結果},
label={fig:stat09}]
{./context/chp4/統計09.png}

\InsertImage
[align = center,scale=0.5, caption={圖 4-10 統計結果},
label={fig:stat10}]
{./context/chp4/統計10.png}


\EndChapter
