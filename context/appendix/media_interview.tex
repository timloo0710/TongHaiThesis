% ------------------------------------------------
\StartChapter{媒體採訪g0v共筆摘錄}{appendix:採訪共筆}
% ------------------------------------------------

g0v成立以來,接受各種媒體(雜誌,報紙,電台…)採訪,\\
 通常接受採訪時,記者朋友們將需求貼在google郵件論壇\\
 上,會有成員在Hackpad上開個共筆區,幾年下來,累積了\\
 不少問與台,這邊摘錄一些\textbf{具代表性的問答記錄}\\
 從貢獻者的回答中,可以更新清楚看出成員本身自己深思\\
 \textbf{社群}的組成及核心價值觀. \\
\section{新新聞【政治事】三二四占領政院要角真的進駐行政院}

  2013 年 5 月\RefBib{website:newnews},clkao 引進德國海盜黨的 Liquid Feedback 線上議決系統,\\
 開啟動民主長期專案,隨後在藍一婷(ETBlue)的主持下成為動民主家族。\\
 這兩年來,除了詳細試用、反覆討論世界各地參與式民主的網路工具外,\\
 也積極開發新的系統,並以「參與實際的政策形成」為主旨,\\
 將心得與連結記錄於動民主 lab 共筆裡。2014 年 12 月 jaclyn \\
 在 g0v 黑客松提案討論時,vTaiwan.tw 專案參與者沿著上述脈絡協助規劃,由此成為 lab 的第四次實驗。
 
 \textbf{小結:}可看出社群內一套軟體產品的演進過程。


\clearpage

% ------------------------------------------------
\section{看雜誌}

5.  你們有提到從反核大遊行到這次的太陽花\RefBib{website:watchmag},讓許多成員有「從雲端到地上」的突破,對於這種轉變和衝擊,最大的感受是什麼?有沒有讓你們當初參與零時政府的想法產生影響或動搖?

hc\textgreater 只是本來大家都是在虛擬的網路上合作,但反核跟這次的學運,反而都是實體的參與。感覺是完全不一樣,但跟原來參與零時政府倒沒太多不同

6.  如何看科技資訊技術對於這次太陽花學運的影響,如果沒有這些新科技的介入,太陽花學運還可以產生這麼大的效應嗎?未來的社會運動,科技的輔助會不會是絕對的必要工具。

au\textgreater 如果沒有網際網路的介入,我想學運的風貌會截然不同,請參考 1980 年代,網際網路出現之前的運動即可瞭解。在沒有網際網路的地區也是有社會運動的,反之亦然,此兩者沒有必然的因果關係。

pofeng\textgreater 網路只是工具,如何運用還是看政府與人民,比如說網路的防火牆技術,一般民間公司是拿來防範入侵,但中國政府是拿來做思想檢查,阻擋人民吸收國外資訊的工作。資訊科技很重要,但是如果只注重科技在社會運動的腳色,而忽略議題倡議者的腳色,可能會重蹈 \"師夷長技以制夷\" 的覆轍。
  
\textbf{小結:}所謂資訊工程師,阿宅,宅男,仍會從虛擬網路走向實際行動,相對於萬人喊讚一人到場的行動力,g0v
在這方面有不一樣的展現。
% ------------------------------------------------
\section{30雜誌}

  一,覺得零時政府是一種媒體或公民組織嗎?或覺得自己的角色是什麼?希望能發揮什麼功效。 \RefBib{website:thirtymag} 
  \\
hc \textgreater 零時政府應該不太能用媒體或公民組織來形容,反而比較像是一個社群或是一個平台。主要在著力於開放資料,並且透過開放政府資料,讓人民更關心我們的政府在做什麼。
kiang \textgreater 零時政府並沒有一個整體共識存在,符合基本原則的專案任何人都可以發起,會看到像是組織的運作只是因為那個專案有比較多的人參與,但即使是參與其中,也沒有什麼絕對的強制性,所以有可能只參與半天、也有可能全程參與。個別專案有設定希望達成的目標,這個目標並不需要經過什麼認證程序。
et \textgreater 對我來說 g0v 是一種開源協作、資訊透明的文化,這個文化在程式領域中已經證實有非常大的具體效益,希望可以推廣到程式以外的地方,讓所有人享受它的好處,尤其是公民行動領域。原因是那系所已經有提供一份樣版出來, 而那份樣版的要求有沒有跟本模板一樣設計, 這個就不作詳細分析. 故如果已經有樣版, 那我就會自動把它們分類成\textit{無法使用這本模板}比較好, 但如果分類錯誤, 請告知.

 四,做這些專案,有偷用上班的時間來做嗎?或就是下班後?參與過程及心境可否描述一些。有沒有些故事可以分享。
hc \textgreater 除了有幾次因為媒體報導而讓一些民眾了解零時政府在做什麼樣的專案之外,有幾次也跟政府單位直接溝通,也獲得一些善意的回應,這是讓人能有些振奮的。
kiang \textgreater 學運期間看到政府粗暴的對待激起了一些危機意識,所以在參與的時間已經開始影響正式工作,甚至有想要將工作中的業務與這裡產生連結的想法,畢竟現在走民主倒車的作法隨時可能剝奪你我現在的自由。
et \textgreater 「偷用上班時間做」這種說法暗示了參與 g0v 跟工作是互斥的,但實際上參與 g0v 開發的工具常常可以在工作時用上,兩者是互相增益的。另,現代的工作型態許多是遠距的,並沒有明確的上下班時間,所以更難有所謂「偷用上班時間做」的情況發生。


六,平日你們會見面嗎?除了黑客松工作坊之外,彼此會認識彼此嗎?平常都是靠網路在聯絡溝通的是嗎?有輕鬆的實體碰面嗎?
hc \textgreater 偶爾會有某些專案的小松,或臨時約的見面。但大多數的溝通還是透過網路,有些人彼此認識,有些則否,但這些並非重點。
kiang \textgreater 很多參與的人並沒有實際碰過面,各種松的存在就是提供了一些實體交流機會,但還是有許多人只出現在網路上。

十一,非技術類的人才在零時政府裡會被打壓嗎?有技術掛帥的現象嗎。
hc \textgreater 在零時政府裡,基本上就是專案的發起跟實作。所以只要願意發起或參與符合基本概念的專案 (open source and/or open content) 就可以參與,在這裡沒有人可以強迫任何人做任何事,所以應該也不會有所謂打壓的事。
kiang \textgreater 零時政府基本上是 \"有貢獻的人為大\" ,這裡的大也不是一種地位上的差別,只是有實質貢獻的人聲音往往容易受到重視,或吸引比較多的目光。目前一些知名的專案的確都是環繞著技術,但並不會有 "擁技術自重" 的情況。
et \textgreater 「打壓」是建立在「有個核心階級可以決定權力和資源的分配」的前提下,才會產生的詞彙。同樣地,社群就是一群同好,沒有階級之分,也沒有權力核心,每個人各行其是,自然也不可能出現打壓這種現象。

十三,有看過30雜誌嗎?對我們有何建議及指教。
au \textgreater 之前沒有看過30雜誌的內文。要說建議的話,我想以上 2, 4, 5, 7, 8, 11 等題的用詞,往往是從社會的支配結構,與個人在結構裡的競爭/競爭力、分眾、異化、位差等角度著眼。由於 g0v 的組成,是建立在開源模式「任何人有想法,都可以自行分支新專案」的網絡互酬文化基礎上,所以在貴雜誌的預設上可能難以著墨太多。如果對當代網絡文化的這個面向有興趣,可以參考黑客倫理與數位時代的書單。
et \textgreater 同意樓上,其實很多問題從出發點開始就很難回答……我有逐條解釋了 XD 有,我有看過 30 雜誌,不過是很多年以前的事情了,所以也沒辦法給出什麼具體的建議…(抓頭)就個人經驗,會想看 30 雜誌的人,多半是處在對未來感到迷惘、想要從別人身上獲取建議、想要尋找(主流媒體下的)典範來模仿的心境下,至少當時的我是如此。近年來因為網路的發展,情況已經與多年前我看 30 時相差甚遠,現在的 30 讀者們,可能需要的不再是提供可以模仿的典範、或者給予符合媒體主流價值的人生建議,因為當前職場的變化速度,已經不是任何一套固定的作法可以無限套用的,現在的 30 讀者們需要的可能是探索自我、啟迪心靈、鍛鍊思辨,練習以哲學的層次來解決問題,才能面對接下來的千變萬化的人生以及職場。

\textbf{小結:}提問者理解g0v的方式,和其他提問者不同,所以問題的用辭有比較有機會讓受訪者討論,是較少見的一篇採訪,
而觸及的問題,比較生活化,看得到社群成員一般生活時的樣貌。

\section{台大意識報}

Question\#1: 您好 \RefBib{website:ntusense} ,我在這次學運有多數的時間是透過觀看「http://g0v.today/」來取得現場狀況的。例如之前在青島東路的「民主講堂」以及立法院內這22天以來的狀況;其他包括文字直播等等。幾乎每天一有空閒就盯著這個Live系統,即使不再現場,依舊關注著最新消息。我想有許多學生、社會民眾也是透過g0v貴單位架設的直播系統平台來關注第一手資訊。想請問這樣的平台系統的想法是從什麼時候開始的呢?在這一次的學運中,又安排了多少人力和什麼樣的人去維護它呢?謝謝~

au \textgreater 這個平台從 2013 年初上線 (詳見此文),於 2013 年 3 月底以 hackfoldr.org 為名開放給公眾使用。此次學運的 http://hackfoldr.org/congressoccupied/ 分區由 clkao 於 3/18 發起,3/22 由 isacl 設定為現行網域名稱 g0v.today,之後在 3/25 見報前 au 重寫後端架構以因應流量,前端頁面則由 yhsiang、swem、cactis、youchenlee 陸續補強功能。內容更新是以試算表匿名共筆方式進行,目前約有 30 人參與內容維護及直播網址更新。

Michael\_LI \textgreater 補充:直播備份組,因為本人覺得直播現場影像,是非常重要的歷史紀錄(文物?),於是3/19號晚上到現場之前先買了一顆硬碟,進行備份影片以及整理列表的工作,不是很顯眼的工作就是了。
-->3月19日就開始有人到現場輪班(青島東路),每個人帶了自己的筆電,就進行了YouYube直播了

lanf0n \textgreater 容易產生誤會的部份是,g0v.today 是比較偏向平台性質,並非所有在 hackfoldr 內的 link 都是 g0v 的專案。

Question\#2: 從其他的採訪中得知g0v是個Open Source的平台,沒有總指揮。除非是特定的專案,可能會有些許的領導部分。這一點我有點疑惑,那要如何判定說今天發起的專案是合乎G∅V的宗旨和不會引起大量爭議且是可以被大眾接受的呢?

au \textgreater g0v 專案只有「源碼開放授權、內容創用CC共享」這個公約數。通常放在 *.g0v.tw 網域的,
是以公部門資料透明化,以及公民協同資訊平台為主,是以此類專案較容易為人接受。
如果有太鮮明的黨派立場,很自然會限縮加入的人數。如遇爭議時,
常會以「技術平台部份留在 g0v,私有應用加值部份在此基礎上另行維護」方式處理,此為粗略共識的協同文化,並不是由上而下的指揮系統達成。


\textbf{小結:}本篇提到源碼,內容的公約數,公部門資料透明化,粗略共識的協同文化。此定義可和其他採訪交互參照。

\section{今週刊}

建立隨時離開都有人接手的文化 \RefBib{website:nowmag}
福利請聽的網站座右銘是「一個成功的社會並不是看富裕的人過得多優渥,而是弱勢者如何自在地生存。
這次的黑客松是第二次在台南舉辦。
最終目的是讓資訊更透明便利,…,不單指供給資訊,也包括資訊的可靠性,
看不下去就自己動手改變

\textbf{小結:}忠實呈現主要核心思想。

\section{g0v官方版的媒體報導}  

媒體報導 \RefBib{website:g0vOfficial}

\subsection{2015 第二季}

\begin{itemize}
\item 三二四占領政院要角真的進駐行政院 新新聞 張家豪 2015.05.13
\item 不想再被當白痴:零時政府要做透明、開源的先鋒 邱柏鈞、鄭婷宇 2015.04.29
\end{itemize}

\subsection{2015 第一季}

\begin{itemize}
\item 教育部於14日發表2015智慧生活創新創業育成平臺計畫成果 新浪新聞 2015.03.14
\item 別只當酸民! 行政院力邀「婉君」參與線上會議建言 NOWnews 2015.03.11
\item 產官學合作催生雲端災防 聯合新聞網 2015.03.08
\item 加強網路溝通 張善政盼與g0v協作 風傳媒 2015.01.09
\end{itemize}

\subsection{2014 第四季}

\begin{itemize}
\item 「鍵盤柯南」化繁為簡 政府資訊變透明 自由時報 2014.12.23
\item 「議員投票指南」 讓你輕鬆了解投票人選 / 蘋果日報 2014.11.25
\item 九合一選舉怎麼投?網路分析工具、政見指南整理 / PunNode 2014.11.24
\item 減少盲目投票!「議員投票指南」公開議員績效 / 自由時報 2014.11.19
\item 網友架平台 公開議員績效 / 中時電子報 2014.11.11
\item Taiwan's g0v: Using Open-Source Code And Communities To Engage Citizens And Make Government More Open / techdirt 2014.11.11
\item 來自不同國家,擁抱相同的開源精神——g0v 零時政府 2014 年會精彩落幕 / Inside 2014.11.11
\item 接軌國際 台零時政府年會精彩落幕 / 中時電子報 2014.11.10
\item 接軌國際、跨界交流 g0v.tw 台灣零時政府年會精彩落幕 / TechNews 2014.11.10
\item g0v首度舉辦大型年會,多國開放資料社群跨界交流 / iThome 2014.11.09
\item After Sunflower Movement, Taiwan's g0v Uses Open Source to Open the Government / TechPresident 2014.11.05
\item 黑客、草根民主与乌托邦—台湾零时政府g0v / 泡泡网报道 2014.11.04
\item 【沃草】黑心油無所遁形!g0v優化政府油品進出口資訊介面 / 蘋果日報 2014.10.26
\item 【沃草】寫程式監督國會!揪出神隱立委、白賊政客 / 蘋果日報 2014.10.08
\item 佔中和平抗命 g0v零時政府隔海助威 / 自由時報 2014.10.01

\end{itemize}

\subsection{2014 第三季}

\begin{itemize}
\item IT人支援香港占中運動大作戰,臺灣g0v也技術支援 / iThome 2014.09.30
\item 結合Google地圖 巢運推「天龍特公地」 / 新頭殼 2014.09.25
\item 病床數不用現場看 網友統整公開便民 / TVBS 2014.08.12
\item g0v打造急診室即時資訊平臺,告訴你傷患往哪送不用等? / IThome 2014.08.12
\item 緩解急診塞車 g0v推全台急診即時看板 / 新頭殼 2014.08.12
\item 政府要汗顏了… g0v推全國「急診」即時看板 / ETToday 2014.08.12
\item 急診塞不塞 「即時看板」全都露 / 聯合報 2014.08.12
\item 相關報導: 急診非先來後到 衛福部:將建更便利平台 / 聯合報 2014.08.12
\item 急診有多忙? 公民自製全國急診即時看版 / 自由時報 2014.08.11
\item 【沃草】g0v推全台急診即時看板,各大醫院急診實況一目了然! / 蘋果日報 2014.08.11
\item 等不及官方 網友自製「高雄地下管線圖」 / 蘋果日報 2014.08.06
\item 網路社群 即時指點生路 / 聯合報 2014.08.02
\item 高雄石化物質氣爆,g0v快速集結「高雄氣爆訊息統整」共筆,展現群體智慧 / 數位時代 2014.08.01
\item 高雄氣爆災情網路資源大匯整(持續更新) / iThome 2014.08.01
\item 【數位觀點】讓力量相乘吧! / 數位時代 2014.07.29
\item 【沃草】工程師COSCUP大會師!開放原始碼精神改變社會 / 蘋果日報 2014.07.21
\item 【沃草】台灣輸出公民運動網路技術!g0v零時政府助陣香港七一佔中 / 蘋果日報 2014.07.02
\end{itemize}

\subsection{2014 第二季}

\begin{itemize}
\item【沃草】政治獻金法成收賄解套法 民團籲修法 / 蘋果日報 2014.06.25
\item 台灣/台南:「沒有人」獲得台南黑客松大獎 / PunNode 2014.06.24
\item 全球最活躍的開放政府組織在台灣! / 蘋果日報 2014.06.13
\item 張善政:市場調節 4G會降價 / 聯合報 2014.06.13
\item 觀察/恐龍般的國民兩黨網路政策想像 / NOWnews 2014.06.10
\item 「零時」時代的「小額參與」:新時代NGO面臨的公民力量 / 國家文化藝術基金會線上誌 2014.06
\item 零時政府,發揮公民的力量 /《30》雜誌 2014年6月號
\item 太陽花運動:數位時代的新社會運動 / 科技濃湯 2014.05.23
\item Diseñando las protestas taiwanesas de \#CongressOccupied / Global Voices 2014.05.10
\item 選後數百筆小額支出 朱立倫駁走路工 / 年代新聞 2014.05.09
\item 「自己的媒體自己架」 網路瘋傳直播教學 / 蘋果日報 2014.05.06
\item 希望幫更多人 羅佩琪設「病後人生」網教申請醫療補助 / ETtoday 2014.05.05
\item 台灣:不要浪費g0v / PunNode 2014.04.30
\item 我們不需要酸民,也不需要超級英雄——「科技創造新世代公民運動」座談紀實 / Inside 2014.04.28
\item 網友認領建檔 政治獻金全透明 / 聯合新聞網 2014.04.26
\item 追政治獻金 十萬鄉民十萬軍 / Yahoo奇摩(新聞)2014.04.23
\item 金錢政治無所遁形!g0v.tw 開放政治獻金紀錄,運用群眾智慧比對資料 / TechNews 2014.04.22
\item How Technology Revolutionized Taiwan’s Sunflower Movement / The Diplomat 2014.04.15 (翻譯)
\item 學運文字現場轉播 與聽障關聯密切 / 公視新聞網 2014.04.10
\item 零時政府4/19辦黑客松聚會 公民議題不退場 / 蘋果日報 2014.04.10
\item 人民渴求的不是評論 而是見證 / 新新聞 2014.04.08 (全文備份)
\item 學運背後的IT推手:g0v零時政府 / iThome 2014.04.08
\item 從學運IT應用看數位匯流發展 / 經濟日報╱社論 2014.04.08
\item 網路大串連,全面引爆服貿話題 / 數位時代 2014.04.07
\item 林佳龍:學運成功運用雲端網路 值得借鏡 / 自立晚報 2014.04.03
\item 史上最大學運推手:免費網路工具 / 商業周刊 2014.04.02
\item 你被服貿了嗎? 秒查各行業怎麼被影響 / 聯合報 2014.04.02
\end{itemize}


\subsection{2014 第一季}

\begin{itemize}
\item 「零時政府」資訊網避壟斷 / 蘋果日報 2014.03.31
\item 挺學運,IT人站出來 / iThome電腦報周刊 2014.03.31
\item 文宣戰/紐時廣告兩天 架英文版網站 / 聯合報 2014.03.30 (全文備份)
\item 新的開始:反服貿(自由貿易協定),青年運動與下個世代的媒體 / 破週報 2014.3.27
\item 三動作護台灣 學生要立委支持先立法再審查 / 蘋果日報 2014.03.26
\item 激情抗爭!佔領立法院背後的科技支援運用 / TechNews 科技新報 2014.03.20
\item 零時政府 用鍵盤革命改造社會 / 今周刊 2014.03.20
\item 政治獻金往哪跑? 投票指南當參考 / 聯合新聞網 2014.03.15
\item 共同抵制假新聞 拒當「謠言」受害者 / PeoPO 公民新聞 2014.03.11
\item 福利請聽 福利資訊的入口網站 / 生命力新聞 2014.03.09
\item g0v 與「國會無雙」(公民一九八五 網路鄉民串聯 25萬民眾自發上街) / 自由時報 2014.02.03
\item 急需補助 「福利請聽」免費查 / 聯合報 2014.01.30 (全文備份)
\item 網路新勢力 顛覆全世界 / 財訊雙週刊 2014.01.28
\item
{
 他們他們正在做!900鄉民用鍵盤捍衛正義 / 商業周刊 1367 期 2014.01.22 (全文備份)
 商業周刊 1367 期專題報導 :談希臘、德國、芬蘭、與 g0v
 延伸閱讀: 民怨商品化?從德國國會觀察網站談起
}
\item 台語輸入法APP 法籍學生研發 / 自由時報 2014.01.12
\end{itemize}


\subsection{2013 第四季}

\begin{itemize}
\item 平民英雄百人榜〉透過鍵盤改變社會「g0v.tw」/ 遠見雜誌 2013.12 (330 期)
\item 【溫肇東專欄】一群勇敢追夢的台灣年輕人 / 經理人月刊 2013.12
\item 投票指南網站 缺席立委、叛逆立委現形 / TVBS 2013.12.27
\item 駭客鄉民 用程式碼「拆」政府 / 天下雜誌 538 期 2013.12.25 ( 全文備份 )
\item 王金平成饒舌歌手? 網友KUSO自律RAP / 蘋果日報 2013.12.17 ( 此新聞與 g0v 的關係 )
\item 萌典,教育部辭典糾察隊! / 聯合晚報 2013.12.7
\item 網友力量大 揪教部辭典上千錯 / 中央日報 2013.11.23
\item g0v 要用鍵盤拆政府搞革命 / 新新聞 2013.11.19 ( 粉絲頁全文 )
\item 立院影城開演 不滿質詢丟鞋抗議 / 台視新聞 2013.11.6
\item 立院影城開幕!不滿問政…鍵盤丟鞋 / 聯合報 2013.11.6
\item 立院影城開演 看不爽!鍵盤丟鞋砸蛋 / 蘋果日報 2013.11.6
\item Open Data 啟動產業新契機 創造人民有感未來生活 六大專家找出新應用 / 經濟日報 2013.10.14
\item Open Data 啟動產業新契機 創造人民有感未來生活 / 大成報
\item Open Data 啟動產業新契機 創造人民有感未來生活 / 行政院研考會 2013.10.7
\item 與零時政府對談分散協作的實力與潛力 / 深音網路廣播訪談逐字稿 2013.10.13
\end{itemize}


\subsection{2013 第三季}

\begin{itemize}
\item TechTalk@TW 零時政府 g0v 專訪 / TechTalk@TW 2013.09.05
\item 靠政府?不如靠自己 – g0v.tw 人民力量大匯集 / 北美智權報 第91期 2013.09.02
\item 鍵盤革命! 高手寫程式監督政府 / 聯合報 2013.08.20 ( 註:「台大資工系」「參與人數近五百人」「這不是偽裝」,詳情請見說明 )
\item 您也是假新聞的受害者嗎?試試「新聞小幫手」 / Inside 2013.08.20
\item 用開放改造社會是開源人的浪漫──開源人年會 / NPOst 週報 No. 45 2013.08.05
\item 工程師的鍵盤革命:拆政府,原地重建 / Inside 2013.08.05
\item 反黑箱服貿協議能量逐漸匯聚 洪案激起公民行動長期經營的反思 / NPOst 週報 No. 44 2013.07.28
\end{itemize}

\subsection{2013 第一、二季}

\begin{itemize}
\item 「網路全黑日」發動 多個網站熄燈封網抗議 (萌典網站加入抗議) / 新頭殼 2013.06.04
\item 佛心求職程式 可揪血汗公司 / 蘋果日報 2013.04.14
\item 「揭露22K」網 揪出血汗企業 / 蘋果日報 2013.01.29
\end{itemize}


\section{零時政府宣言}

g0v 宣言 \RefBib{website:manifesto} 

\subsection{我們來自四方}

g0v.tw 是一個致力於打造資訊透明化的社群。g0v.tw 的參與者來自四方,有程式開發者、設計師、社會運動工作者、教育工作者、文字工作者、公民與鄉民等來自各領域的人士。這些人聚在一起,希望資訊透明化可以更進一步的改善台灣的公民環境。只要有心想用自己的專業及能力來參與,就可以加入 g0v.tw。

\subsection{我們支持言論自由、資訊開放}

g0v.tw 以開放原始碼的精神為基底,關心言論自由、資訊開放,希望可以最新的科技,提供讓公民更容易使用的資訊服務。資訊的透明化可以幫助公民更確實了解政府運作,更快速了解議題,更有效監督政府,確保政府不脫離民有、民治、民享的本質。

\subsection{我們自主參與,成果開放}

我們平時透過 g0v.tw 各網路平台(IRC, hackpad, github)溝通協作,或參與不定期舉辦的黑客松活動。我們的成果(包括文件、程式碼、運算資料、數據分析結果及過程執行方式)需遵循開放原始碼授權,讓更多人能使用、改善、回饋,發揮最大效用。各專案成果不屬於 g0v.tw,但也歡迎在此平台共享。

\subsection{我們很歡樂,也想改變現狀}

我們喜歡找到問題,樂於討論解決方案,願意動手嘗試解決問題。我們在多元領域中找到合作的途徑,讓力量相乘,以想像力指引新的方向。我們希冀以行動改變現狀,不想淪為沉默的幫凶。

\subsection{我們去中心化,實踐流動民主}

g0v.tw 沒有負責人、代言人,由參與者自主決定想要參與的專案,同時加深 g0v.tw 的社群文化。各專案各自運作討論決策,g0v.tw 社群平台相關重要議題則使用流動式網路民主系統討論與決策。

\subsection{如果你想看到特定專案加速進行}

g0v.tw 無黨無派、無錢通買菜,是草根集結的公民運動,你可以參與專案贊助腦力、勞力,也可以捐款贊助舉辦黑客松、或直接支持特定專案。

\subsection{我們就是你}

如果你認同以上,歡迎加入 g0v.tw,來聊聊你想作些什麼、想協助什麼專案、想看到我們身處的世界有什麼改變。歡迎你成為科技改變社會的力量。

\section{零時政府宣言英文版}

g0v MANIFESTO \RefBib{website:manifesto}

\subsection{From across Taiwan}

g0v.tw is a community that advocates transparency of information, also known as open data. We are passionate coders, designers, activists, educators, writers and citizens from across Taiwan. Through working together to bring data into the open, we hope to build a better Taiwan for its citizens. To join g0v, all you need is to be ready and willing to use your expertise or energy for our cause.

\subsection{Freedom of Speech and Information Transparency}

Built on the spirit of the open-source community, g0v stands for freedom of speech and information transparency. We aim to use technology in the interest of the public good, allowing citizens easy access to vital information. Opening up and making data public allows the people of Taiwan to take a closer look at politics and important issues. This gives them the tools needed to evaluate their government and exert their democratic right to decide how politicians act.

\subsection{Independent and Transparent}

We cooperate through our website, g0v.tw, and through open platforms, such as IRC, hackpad and github, among others. Our work is also done through open events, such as our “hackathons”. Everything we produce — source code, documents, formulas, analyses and processes — is made available for anyone to view, react to, use and improve.

\subsection{Sanguine on Fostering Change}

At g0v, we love to seek out problems and explore solutions. Further, we are willing to take action to implement these solutions. Through various types of cooperation, we multiply the impact and creativity of our ideas. We aim to bring about change and are not willing to resort to cynicism or apathy.

\subsection{Liquid Democracy}

g0v doesn’t have a leader or spokesperson. We are a liquid democracy. Participants decide what to work on and are at the heart of our communal culture. Using Liquid Democracy techniques, our community makes decisions on important issues and strategic direction.

\subsection{If you’re interested, Join us}

Our community is grassroots: we are non-partisan, unbiased and not for sale. You can contribute to our hackathons with your skills, ideas, time or donations, or provide support for specific projects.

\subsection{You’re Next}

If you like what you’ve read, head to g0v.tw to share your project ideas, offer help or let us know how you want society to change. We welcome you to join us in using technology to change society for the better.

% ------------------------------------------------
\EndChapter
% ------------------------------------------------
